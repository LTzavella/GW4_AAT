\documentclass[man]{apa6}
\usepackage{lmodern}
\usepackage{amssymb,amsmath}
\usepackage{ifxetex,ifluatex}
\usepackage{fixltx2e} % provides \textsubscript
\ifnum 0\ifxetex 1\fi\ifluatex 1\fi=0 % if pdftex
  \usepackage[T1]{fontenc}
  \usepackage[utf8]{inputenc}
\else % if luatex or xelatex
  \ifxetex
    \usepackage{mathspec}
  \else
    \usepackage{fontspec}
  \fi
  \defaultfontfeatures{Ligatures=TeX,Scale=MatchLowercase}
\fi
% use upquote if available, for straight quotes in verbatim environments
\IfFileExists{upquote.sty}{\usepackage{upquote}}{}
% use microtype if available
\IfFileExists{microtype.sty}{%
\usepackage{microtype}
\UseMicrotypeSet[protrusion]{basicmath} % disable protrusion for tt fonts
}{}
\usepackage{hyperref}
\hypersetup{unicode=true,
            pdftitle={Effect of inhibitory control training on food-related action tendencies, liking and impulsive choice},
            pdfauthor={First Author~\& Ernst-August Doelle},
            pdfkeywords={keywords},
            pdfborder={0 0 0},
            breaklinks=true}
\urlstyle{same}  % don't use monospace font for urls
\usepackage{graphicx}
% grffile has become a legacy package: https://ctan.org/pkg/grffile
\IfFileExists{grffile.sty}{%
\usepackage{grffile}
}{}
\makeatletter
\def\maxwidth{\ifdim\Gin@nat@width>\linewidth\linewidth\else\Gin@nat@width\fi}
\def\maxheight{\ifdim\Gin@nat@height>\textheight\textheight\else\Gin@nat@height\fi}
\makeatother
% Scale images if necessary, so that they will not overflow the page
% margins by default, and it is still possible to overwrite the defaults
% using explicit options in \includegraphics[width, height, ...]{}
\setkeys{Gin}{width=\maxwidth,height=\maxheight,keepaspectratio}
\IfFileExists{parskip.sty}{%
\usepackage{parskip}
}{% else
\setlength{\parindent}{0pt}
\setlength{\parskip}{6pt plus 2pt minus 1pt}
}
\setlength{\emergencystretch}{3em}  % prevent overfull lines
\providecommand{\tightlist}{%
  \setlength{\itemsep}{0pt}\setlength{\parskip}{0pt}}
\setcounter{secnumdepth}{0}
% Redefines (sub)paragraphs to behave more like sections
\ifx\paragraph\undefined\else
\let\oldparagraph\paragraph
\renewcommand{\paragraph}[1]{\oldparagraph{#1}\mbox{}}
\fi
\ifx\subparagraph\undefined\else
\let\oldsubparagraph\subparagraph
\renewcommand{\subparagraph}[1]{\oldsubparagraph{#1}\mbox{}}
\fi

%%% Use protect on footnotes to avoid problems with footnotes in titles
\let\rmarkdownfootnote\footnote%
\def\footnote{\protect\rmarkdownfootnote}


  \title{Effect of inhibitory control training on food-related action tendencies, liking and impulsive choice}
    \author{First Author\textsuperscript{1}~\& Ernst-August Doelle\textsuperscript{1,2}}
    \date{}
  
\shorttitle{ICT EFFECTS ON FOOD-RELATED ACTION TENDENCIES, LIKING AND CHOICE}
\affiliation{
\vspace{0.5cm}
\textsuperscript{1} Wilhelm-Wundt-University\\\textsuperscript{2} Konstanz Business School}
\keywords{keywords\newline\indent Word count: X}
\usepackage{csquotes}
\usepackage{upgreek}
\captionsetup{font=singlespacing,justification=justified}

\usepackage{longtable}
\usepackage{lscape}
\usepackage{multirow}
\usepackage{tabularx}
\usepackage[flushleft]{threeparttable}
\usepackage{threeparttablex}

\newenvironment{lltable}{\begin{landscape}\begin{center}\begin{ThreePartTable}}{\end{ThreePartTable}\end{center}\end{landscape}}

\makeatletter
\newcommand\LastLTentrywidth{1em}
\newlength\longtablewidth
\setlength{\longtablewidth}{1in}
\newcommand{\getlongtablewidth}{\begingroup \ifcsname LT@\roman{LT@tables}\endcsname \global\longtablewidth=0pt \renewcommand{\LT@entry}[2]{\global\advance\longtablewidth by ##2\relax\gdef\LastLTentrywidth{##2}}\@nameuse{LT@\roman{LT@tables}} \fi \endgroup}


\DeclareDelayedFloatFlavor{ThreePartTable}{table}
\DeclareDelayedFloatFlavor{lltable}{table}
\DeclareDelayedFloatFlavor*{longtable}{table}
\makeatletter
\renewcommand{\efloat@iwrite}[1]{\immediate\expandafter\protected@write\csname efloat@post#1\endcsname{}}
\makeatother
\usepackage{lineno}

\linenumbers
\usepackage{tikz}
\usepackage{xcolor}

\hypersetup{ colorlinks, linkcolor={red!60!blue}, citecolor={blue!50!black}, urlcolor={blue!80!black} } \urlstyle{same}
\usepackage{multicol}
\usepackage{enumitem}

\setlist{nosep}
\usepackage[font=footnotesize,labelfont=bf]{caption}
\usepackage{amsmath}
\usepackage{threeparttable}
\usepackage{soul}
\usepackage{caption}
\usepackage{booktabs}
\usepackage{float}

\authornote{{[}Add complete departmental affiliations for each author here. Each new line herein must be indented, like this line.{]}

{[}Enter author note here.{]}

Correspondence concerning this article should be addressed to First Author, Postal address. E-mail: \href{mailto:my@email.com}{\nolinkurl{my@email.com}}}

\abstract{
{[}ADD ABSTRACT HERE{]}


}

\begin{document}
\maketitle

\hypertarget{introduction}{%
\section{Introduction}\label{introduction}}

\par

The recent rise in overweight and obesity rates can primarily be attributed to the over-consumption of energy-dense foods that are high in fat, sugar and salt content (World Health Organization, 2018). One theoretical explanation for overeating has been provided by dual-process models which argue that behaviour is determined by the interaction of impulsive (\emph{automatic}) and reflective (\emph{controlled}) processes (Kakoschke et al., 2015; Strack \& Deutsch, 2004). For example, over-consumption of unhealthy foods can be ascribed to heightened automatic biases for such foods, which can result in increased food intake if these automatic tendencies are not regulated via controlled processes (Kakoschke et al., 2017b).\(\color{red}{\text{provide example here – e.g. exposure to unhealthy foods in the environ (advertising)}}\)

\(\color{red}{\text{can trigger biased behavior and result in consumption for those who have limited self-control (either trait or state!}}\)

\par

Theoretical frameworks, such as the dual process model, have led to the development of behaviour change interventions for unhealthy eating behaviours that target either automatic or controlled processing, such as approach bias modification and inhibitory control training respectively (see Kakoschke et al., 2017a; Jones, Hardman, Lawrence, \& Field, 2018 for recent reviews). \(\color{red}{\text{Need to take a step back here and explain ICT- Suggest discussion of inhibition, how it’s}}\)
\(\color{red}{\text{related to food-related behavior , training and associated effects on behavior. Discuss success in terms of }}\)
\(\color{red}{\text{findings and meta-analyses. Provide some key examples from the literature}}\)

\par

\(\color{red}\text{{Rachel's comment below:}}\)
After discussing ICT and effects on behavior etc. I think we should discuss mechanism of devaluation\ldots{} and then link to approach/ avoid- Also need to provide some discussion of increased preferences for go foods\ldots{} whilst appreciating the point that we do not have a cued-approach design (so there are some differences) but the logic / theory still applies

\(\color{red}\text{{CUT bit:  }}\)
The primary aim of the present study was to investigate the interaction between automatic and controlled processing in the context of inhibitory control training (ICT). Specifically, we tested whether food-specific ICT could influence individuals' automatic action tendencies towards unhealthy foods.

\par

This study focuses on an automatic process known as approach bias, which is the automatic action tendency to approach an appetitive (food) cue in the environment, rather than avoid it (C. E. Wiers et al., 2013). An approach bias has been demonstrated for a variety of energy-dense foods in both obese and healthy-weight individuals (Brignell, Griffiths, Bradley, \& Mogg, 2009; Kemps \& Tiggemann, 2015; Kemps, Tiggemann, Martin, \& Elliott, 2013; Veenstra \& de Jong, 2010). Interestingly, Kakoschke et al. (2015) found that approach bias alone did not predict increased intake of unhealthy foods, but it was the interaction between approach bias and inhibitory control that was the significant determinant of subsequent behaviour. The authors report that approach bias had the expected effect on food intake only for participants with low inhibitory control. An important component of controlled processing is inhibitory control, which refers to the ability to inhibit behaviours and impulses that are incompatible with higher-order goals, such as wanting to lose weight (Houben, Nederkoorn, \& Jansen, 2012), and encompasses several elements, such as response inhibition and cognitive flexibility (see Bartholdy, Dalton, O'Daly, Campbell, \& Schmidt, 2016). Inhibitory control capacity is often measured via response inhibition paradigms, such as the go/no-go task (Donders, 1969; Newman \& Kosson, 1986) and stop-signal task (Lappin \& Eriksen, 1966; Logan, Cowan, \& Davis, 1984), and has been associated with unhealthy eating behaviours (e.g., Jasinska et al., 2012; Guerrieri et al., 2007; Hall, 2012). For example, Nederkoorn, Houben, Hofmann, Roefs, and Jansen (2010) showed that strong implicit preferences for snacks paired with low \enquote{inhibitory control capacity} predicted weight gain over one year. Overall, there is evidence to suggest that both inhibitory control and motivational processes are important determinants of eating-related behaviour.

\par

Complementary evidence for the role of automatic and controlled processes in the regulation of eating behaviours stems from the line of research dedicated to the development of health behaviour change interventions. Approach bias modification training is commonly delivered via an approach-avoidance task (AAT; Neumann \& Strack, 2000; Rinck \& Becker, 2007; R. W. Wiers et al., 2013) and has been found to be effective in re-training approach bias for foods (Brockmeyer, Hahn, Reetz, Schmidt, \& Friederich, 2015) and even reduce food intake in the laboratory (Schumacher, Kemps, \& Tiggemann, 2016; see Kakoschke et al., 2017a for review). The AAT is assumed to capture automatic action tendencies when participants are instructed to respond to task-irrelevant feature such as the orientation (portrait or landscape) of the presented picture, by pulling or pushing a joystick (C. E. Wiers et al., 2013). The AAT can also pair actions with visual feedback, so that the picture gets bigger when participant pull the joystick towards them (zoom-in) and gets smaller when they push it away (zoom-out). Arm extension could indicate an approach response towards an appetitive food (object-reference) or an avoidance response when the food is pushed away from the body/self (Phaf, Mohr, Rotteveel, \& Wicherts, 2014) and thus visual feedback provides the self-reference attribute to the responses (e.g., object comes closer to one's body). The \enquote{zooming} feature disambiguates the mapping of responses to approach and avoidance actions, whereby pulling the joystick represents approach and pushing it reflects avoidance (Neumann \& Strack, 2000). In AAT training, contingencies between actions and stimuli are manipulated so that appetitive cues are associated with push actions (avoidance) and neutral items are paired with pull actions (approach).

\par

In the context of controlled processes, ICT interventions involve cue-specific go/no-go or stop-signal tasks whereby participants are instructed to make a speeded choice response to appetitive stimuli such as foods or alcohol, but to withhold that response when a visual, or auditory, signal is presented. In go/no-go training, signal-stimulus mappings are manipulated so that appetitive cues (e.g., unhealthy foods) are consistently paired with a stop signal. Stopping to unhealthy foods has been shown to reduce food consumption (Adams, Lawrence, Verbruggen, \& Chambers, 2017; Houben \& Jansen, 2011, 2015; Lawrence et al., 2015; Veling, Aarts, \& Papies, 2011; also see Allom, Mullan, \& Hagger, 2016 for meta-analysis) and promote healthy food choices in the laboratory (Veling, Aarts, \& Stroebe, 2013; Veling, Chen, et al., 2017). ICT protocols have even been associated with increased weight loss (Lawrence, O'Sullivan, et al., 2015; Veling, van Koningsbruggen, Aarts, \& Stroebe, 2014). A potential mechanism of action behind ICT effects on food consumption is stimulus devaluation (Veling et al., 2017), whereby the evaluations of appetitive foods are reduced during training to facilitate performance when response inhibition is required (e.g., Chen, Veling, Dijksterhuis, \& Holland, 2016). A possible explanation for this devaluation effect is provided by the Behaviour Stimulus Interaction (BSI) theory which posits that food stimuli are devalued when negative affect is induced to resolve the ongoing conflict between triggered approach reactions to appetitive foods and the need to inhibit responses towards those stimuli (Chen et al., 2016; Veling, Holland, \& van Knippenberg, 2008; Veling et al., 2017). When a food is devalued, the approach bias towards that cue is reduced and inhibition can successfully take place. It is possible that this reduction in approach bias can be observed after go/no-go training\footnote{It should be noted that a link between affect and motivation is also included in the dual-process framework of (eating) behaviour}.

\par

Theoretically, the effects of ICT could also be explained by hard-wired connections between Pavlovian appetitive/ aversive centres {[}dickinson\_mechanisms\_2014{]} and go/no-go responses (McLaren \& Verbruggen, 2016; Verbruggen, url, Bowditch, Stevens, \& McLaren, 2014). Verbruggen et al. (2014) suggest that conditioned inhibitory control in ICT paradigms, such as go/no-go training, can have an influence not only on the evaluations of stimuli, but also their motivational value via links to the appetitive/aversive centres. For example, a stimulus consistently paired with stopping on signal trials would be devalued and the approach motivation for that stimulus could also be reduced via a hard-wired excitatory connection between the \enquote{stop} system and the aversive centre (also see Dickinson \& Boakes, 2014; McLaren \& Verbruggen, 2016). Similarly, approach bias for foods consistently paired with go responses on no-signal trials could be increased via the link between the \enquote{go} system and the appetitive centre.

\par

This study attempts to answer this question by employing a go/no-go training paradigm with unhealthy food stimuli and measuring automatic action tendencies via an AAT before and after training. We tested whether individuals would show reduced approach bias for the foods associated with response inhibition and/or increased approach bias for the foods paired with go responses after training. Consistent with previous ICT literature, the study also examined impulsive food choice and food liking as secondary training outcomes. The research study was conducted as part of the GW4 Undergraduate Psychology Consortium 2017/2018 and data were collected across Cardiff University, University of Exeter and University of Bath campuses.

\hypertarget{methods}{%
\section{Methods}\label{methods}}

\hypertarget{results}{%
\section{Results}\label{results}}

\hypertarget{discussion}{%
\section{Discussion}\label{discussion}}

\newpage

\hypertarget{references}{%
\section{References}\label{references}}

\begingroup
\setlength{\parindent}{-0.5in}
\setlength{\leftskip}{0.5in}

\hypertarget{refs}{}
\leavevmode\hypertarget{ref-adams_training_2017}{}%
Adams, R. C., Lawrence, N. S., Verbruggen, F., \& Chambers, C. D. (2017). Training response inhibition to reduce food consumption: Mechanisms, stimulus specificity and appropriate training protocols. \emph{Appetite}, \emph{109}, 11--23. \url{https://doi.org/10.1016/j.appet.2016.11.014}

\leavevmode\hypertarget{ref-allomDoesInhibitoryControl2016}{}%
Allom, V., Mullan, B., \& Hagger, M. (2016). Does inhibitory control training improve health behaviour? A meta-analysis. \emph{Health Psychology Review}, \emph{10}(2), 168--186. \url{https://doi.org/10.1080/17437199.2015.1051078}

\leavevmode\hypertarget{ref-bartholdySystematicReviewRelationship2016}{}%
Bartholdy, S., Dalton, B., O'Daly, O. G., Campbell, I. C., \& Schmidt, U. (2016). A systematic review of the relationship between eating, weight and inhibitory control using the stop signal task. \emph{Neuroscience \& Biobehavioral Reviews}, \emph{64}, 35--62. \url{https://doi.org/10.1016/j.neubiorev.2016.02.010}

\leavevmode\hypertarget{ref-brignell_attentional_2009}{}%
Brignell, C., Griffiths, T., Bradley, B. P., \& Mogg, K. (2009). Attentional and approach biases for pictorial food cues. Influence of external eating. \emph{Appetite}, \emph{52}(2), 299--306. \url{https://doi.org/10.1016/j.appet.2008.10.007}

\leavevmode\hypertarget{ref-brockmeyer_approach_2015-1}{}%
Brockmeyer, T., Hahn, C., Reetz, C., Schmidt, U., \& Friederich, H. C. (2015). Approach Bias Modification in Food Craving - A Proof-of-Concept Study. \emph{European Eating Disorders Review}, \emph{23}(5), 352--360. \url{https://doi.org/10.1002/erv.2382}

\leavevmode\hypertarget{ref-chen_how_2016}{}%
Chen, Z., Veling, H., Dijksterhuis, A., \& Holland, R. W. (2016). How does not responding to appetitive stimuli cause devaluation: Evaluative conditioning or response inhibition? \emph{Journal of Experimental Psychology: General}, \emph{145}(12), 1687--1701. \url{https://doi.org/10.1037/xge0000236}

\leavevmode\hypertarget{ref-dickinson_mechanisms_2014}{}%
Dickinson, A., \& Boakes, R. A. (2014). \emph{Mechanisms of Learning and Motivation: A Memorial Volume To Jerzy Konorski}. New York, NY: Psychology Press.

\leavevmode\hypertarget{ref-dondersSpeedMentalProcesses1969}{}%
Donders, F. C. (1969). On the speed of mental processes. \emph{Acta Psychologica}, \emph{30}, 412--431. \url{https://doi.org/10.1016/0001-6918(69)90065-1}

\leavevmode\hypertarget{ref-guerrieri_influence_2007}{}%
Guerrieri, R., Nederkoorn, C., Stankiewicz, K., Alberts, H., Geschwind, N., Martijn, C., \& Jansen, A. (2007). The influence of trait and induced state impulsivity on food intake in normal-weight healthy women. \emph{Appetite}, \emph{49}(1), 66--73. \url{https://doi.org/10.1016/j.appet.2006.11.008}

\leavevmode\hypertarget{ref-hall_executive_2012}{}%
Hall, P. A. (2012). Executive control resources and frequency of fatty food consumption: Findings from an age-stratified community sample. \emph{Health Psychology: Official Journal of the Division of Health Psychology, American Psychological Association}, \emph{31}(2), 235--241. \url{https://doi.org/10.1037/a0025407}

\leavevmode\hypertarget{ref-houben_training_2011}{}%
Houben, K., \& Jansen, A. (2011). Training inhibitory control. A recipe for resisting sweet temptations. \emph{Appetite}, \emph{56}(2), 345--349. \url{https://doi.org/10.1016/j.appet.2010.12.017}

\leavevmode\hypertarget{ref-houben_chocolate_2015}{}%
Houben, K., \& Jansen, A. (2015). Chocolate equals stop: Chocolate-specific inhibition training reduces chocolate intake and go associations with chocolate. \emph{Appetite}, \emph{87}, 318--323. \url{https://doi.org/10.1016/j.appet.2015.01.005}

\leavevmode\hypertarget{ref-houben_too_2012}{}%
Houben, K., Nederkoorn, C., \& Jansen, A. (2012). Too tempting to resist? Past success at weight control rather than dietary restraint determines exposure-induced disinhibited eating. \emph{Appetite}, \emph{59}(2), 550--555. \url{https://doi.org/10.1016/j.appet.2012.07.004}

\leavevmode\hypertarget{ref-jasinska_impulsivity_2012}{}%
Jasinska, A. J., Yasuda, M., Burant, C. F., Gregor, N., Khatri, S., Sweet, M., \& Falk, E. B. (2012). Impulsivity and inhibitory control deficits are associated with unhealthy eating in young adults. \emph{Appetite}, \emph{59}(3), 738--747. \url{https://doi.org/10.1016/j.appet.2012.08.001}

\leavevmode\hypertarget{ref-jonesCognitiveTrainingPotential2018}{}%
Jones, A., Hardman, C. A., Lawrence, N., \& Field, M. (2018). Cognitive training as a potential treatment for overweight and obesity: A critical review of the evidence. \emph{Appetite}, \emph{124}, 50--67. \url{https://doi.org/10.1016/j.appet.2017.05.032}

\leavevmode\hypertarget{ref-kakoschke_approach_2017}{}%
Kakoschke, N., Kemps, E., \& Tiggemann, M. (2017a). Approach bias modification training and consumption: A review of the literature. \emph{Addictive Behaviors}, \emph{64}, 21--28. \url{https://doi.org/10.1016/j.addbeh.2016.08.007}

\leavevmode\hypertarget{ref-kakoschke_effect_2017}{}%
Kakoschke, N., Kemps, E., \& Tiggemann, M. (2017b). The effect of combined avoidance and control training on implicit food evaluation and choice. \emph{Journal of Behavior Therapy and Experimental Psychiatry}, \emph{55}, 99--105. \url{https://doi.org/10.1016/j.jbtep.2017.01.002}

\leavevmode\hypertarget{ref-kakoschke_combined_2015}{}%
Kakoschke, N., Kemps, E., Tiggemann, M., Kakoschke, N., Kemps, E., \& Tiggemann, M. (2015). Combined effects of cognitive bias for food cues and poor inhibitory control on unhealthy food intake. \emph{Appetite}, \emph{87}, 358--364. \url{https://doi.org/10.1016/j.appet.2015.01.004}

\leavevmode\hypertarget{ref-kemps_approach_2015}{}%
Kemps, E., \& Tiggemann, M. (2015). Approach bias for food cues in obese individuals. \emph{Psychology \& Health}, \emph{30}(3), 370--380. \url{https://doi.org/10.1080/08870446.2014.974605}

\leavevmode\hypertarget{ref-kemps_implicit_2013}{}%
Kemps, E., Tiggemann, M., Martin, R., \& Elliott, M. (2013). Implicit approachAvoidance associations for craved food cues. \emph{Journal of Experimental Psychology: Applied}, \emph{19}(1), 30--38. \url{https://doi.org/10.1037/a0031626}

\leavevmode\hypertarget{ref-lappinUseDelayedSignal1966}{}%
Lappin, J. S., \& Eriksen, C. W. (1966). Use of a delayed signal to stop a visual reaction-time response. \emph{Journal of Experimental Psychology}, \emph{72}(6), 805--811. \url{https://doi.org/10.1037/h0021266}

\leavevmode\hypertarget{ref-lawrence_training_2015}{}%
Lawrence, N. S., O'Sullivan, J., Parslow, D., Javaid, M., Adams, R. C., Chambers, C. D., \ldots{} Verbruggen, F. (2015). Training response inhibition to food is associated with weight loss and reduced energy intake. \emph{Appetite}, \emph{95}, 17--28. \url{https://doi.org/10.1016/j.appet.2015.06.009}

\leavevmode\hypertarget{ref-lawrence_stopping_2015}{}%
Lawrence, N. S., Verbruggen, F., Morrison, S., Adams, R. C., \& Chambers, C. D. (2015). Stopping to food can reduce intake. Effects of stimulus-specificity and individual differences in dietary restraint. \emph{Appetite}, \emph{85}, 91--103. \url{https://doi.org/10.1016/j.appet.2014.11.006}

\leavevmode\hypertarget{ref-loganAbilityInhibitSimple1984}{}%
Logan, G. D., Cowan, W. B., \& Davis, K. A. (1984). On the ability to inhibit simple and choice reaction time responses: A model and a method. \emph{Journal of Experimental Psychology. Human Perception and Performance}, \emph{10}(2), 276--291.

\leavevmode\hypertarget{ref-mclaren_association_2016}{}%
McLaren, I. P. L., \& Verbruggen, F. (2016). Association, Inhibition, and Action. In R. A. Murphy \& R. C. Honey (Eds.), \emph{The Wiley Handbook on the Cognitive Neuroscience of Learning}. Chichester, England: John Wiley \& Sons, Ltd.

\leavevmode\hypertarget{ref-nederkoorn_control_2010}{}%
Nederkoorn, C., Houben, K., Hofmann, W., Roefs, A., \& Jansen, A. (2010). Control Yourself or Just Eat What You Like? Weight Gain Over a Year Is Predicted by an Interactive Effect of Response Inhibition and Implicit Preference for Snack Foods. \emph{Health Psychology}, \emph{29}(4), 389--393. \url{https://doi.org/10.1037/a0019921}

\leavevmode\hypertarget{ref-neumann_approach_2000}{}%
Neumann, R., \& Strack, F. (2000). Approach and Avoidance: The Influence of Proprioceptive and Exteroceptive Cues on Encoding of Affective Information. \emph{Journal of Personality and Social Psychology}, \emph{79}(1), 39--48. \url{https://doi.org/10.1037//0022-3514.79.1.39}

\leavevmode\hypertarget{ref-newmanPassiveAvoidanceLearning1986}{}%
Newman, J. P., \& Kosson, D. S. (1986). Passive avoidance learning in psychopathic and nonpsychopathic offenders. \emph{Journal of Abnormal Psychology}, \emph{95}(3), 252--256. \url{https://doi.org/10.1037/0021-843x.95.3.252}

\leavevmode\hypertarget{ref-phaf_approach_2014}{}%
Phaf, R. H., Mohr, S. E., Rotteveel, M., \& Wicherts, J. M. (2014). Approach, avoidance, and affect: A meta-analysis of approach-avoidance tendencies in manual reaction time tasks. \emph{Frontiers in Psychology}, \emph{5}(378), 1--16. \url{https://doi.org/10.3389/fpsyg.2014.00378}

\leavevmode\hypertarget{ref-rinck_approach_2007}{}%
Rinck, M., \& Becker, E. S. (2007). Approach and avoidance in fear of spiders. \emph{Journal of Behavior Therapy and Experimental Psychiatry}, \emph{38}(2), 105--120. \url{https://doi.org/10.1016/j.jbtep.2006.10.001}

\leavevmode\hypertarget{ref-schumacher_bias_2016}{}%
Schumacher, S. E., Kemps, E., \& Tiggemann, M. (2016). Bias modification training can alter approach bias and chocolate consumption. \emph{Appetite}, \emph{96}, 219--224. \url{https://doi.org/10.1016/j.appet.2015.09.014}

\leavevmode\hypertarget{ref-strack_reflective_2004}{}%
Strack, F., \& Deutsch, R. (2004). Reflective and Impulsive Determinants of Social Behavior. \emph{Personality and Social Psychology Review}, \emph{8}(3), 220--247. \url{https://doi.org/10.1207/s15327957pspr0803_1}

\leavevmode\hypertarget{ref-veenstra_restrained_2010}{}%
Veenstra, E. M., \& de Jong, P. J. (2010). Restrained eaters show enhanced automatic approach tendencies towards food. \emph{Appetite}, \emph{55}(1), 30--36. \url{https://doi.org/10.1016/j.appet.2010.03.007}

\leavevmode\hypertarget{ref-veling_using_2011}{}%
Veling, H., Aarts, H., \& Papies, E. K. (2011). Using stop signals to inhibit chronic dieters' responses toward palatable foods. \emph{Behaviour Research and Therapy}, \emph{49}(11), 771--780. \url{https://doi.org/10.1016/j.brat.2011.08.005}

\leavevmode\hypertarget{ref-veling_stop_2013}{}%
Veling, H., Aarts, H., \& Stroebe, W. (2013). Stop signals decrease choices for palatable foods through decreased food evaluation. \emph{Frontiers in Psychology}, \emph{4}(875), 1--7. \url{https://doi.org/10.3389/fpsyg.2013.00875}

\leavevmode\hypertarget{ref-veling_training_2017}{}%
Veling, H., Chen, Z., Tombrock, M. C., M. Verpaalen, I. a., Schmitz, L. I., Dijksterhuis, A., \& Holland, R. W. (2017). Training Impulsive Choices for Healthy and Sustainable Food. \emph{Journal of Experimental Psychology: Applied}, \emph{23}(1), 1--14. \url{https://doi.org/10.1037/xap0000112}

\leavevmode\hypertarget{ref-veling_when_2008}{}%
Veling, H., Holland, R. W., \& van Knippenberg, A. (2008). When approach motivation and behavioral inhibition collide: Behavior regulation through stimulus devaluation. \emph{Journal of Experimental Social Psychology}, \emph{44}(4), 1013--1019. \url{https://doi.org/10.1016/j.jesp.2008.03.004}

\leavevmode\hypertarget{ref-veling_what_2017}{}%
Veling, H., Lawrence, N. S., Chen, Z., van Koningsbruggen, G. M., \& Holland, R. W. (2017). What Is Trained During Food Go/No-Go Training? A Review Focusing on Mechanisms and a Research Agenda. \emph{Current Addiction Reports}, \emph{4}(1), 35--41. \url{https://doi.org/10.1007/s40429-017-0131-5}

\leavevmode\hypertarget{ref-veling_targeting_2014}{}%
Veling, H., van Koningsbruggen, G. M., Aarts, H., \& Stroebe, W. (2014). Targeting impulsive processes of eating behavior via the internet. Effects on body weight. \emph{Appetite}, \emph{78}, 102--109. \url{https://doi.org/10.1016/j.appet.2014.03.014}

\leavevmode\hypertarget{ref-verbruggen_inhibitory_2014}{}%
Verbruggen, F., url, M., Bowditch, W. A., Stevens, T., \& McLaren, I. P. L. (2014). The inhibitory control reflex. \emph{Neuropsychologia}, \emph{65}, 263--278. \url{https://doi.org/10.1016/j.neuropsychologia.2014.08.014}

\leavevmode\hypertarget{ref-wiers_automatic_2013}{}%
Wiers, C. E., Kühn, S., Javadi, A. H., Korucuoglu, O., Wiers, R. W., Walter, H., \ldots{} Bermpohl, F. (2013). Automatic approach bias towards smoking cues is present in smokers but not in ex-smokers. \emph{Psychopharmacology}, \emph{229}(1), 187--197. \url{https://doi.org/10.1007/s00213-013-3098-5}

\leavevmode\hypertarget{ref-wiers_should_2013}{}%
Wiers, R. W., Gladwin, T. E., \& Rinck, M. (2013). Should we train alcohol-dependent patients to avoid alcohol? \emph{Frontiers in Psychiatry}, \emph{4}(33), 33. \url{https://doi.org/10.3389/fpsyt.2013.00033}

\leavevmode\hypertarget{ref-whoObesityOverweight2018}{}%
World Health Organization. (2018). Obesity and overweight. Retrieved from \url{https://www.who.int/news-room/fact-sheets/detail/obesity-and-overweight}

\endgroup


\end{document}
