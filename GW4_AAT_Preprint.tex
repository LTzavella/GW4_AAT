\documentclass[man,floatsintext]{apa6}
\usepackage{lmodern}
\usepackage{amssymb,amsmath}
\usepackage{ifxetex,ifluatex}
\usepackage{fixltx2e} % provides \textsubscript
\ifnum 0\ifxetex 1\fi\ifluatex 1\fi=0 % if pdftex
  \usepackage[T1]{fontenc}
  \usepackage[utf8]{inputenc}
\else % if luatex or xelatex
  \ifxetex
    \usepackage{mathspec}
  \else
    \usepackage{fontspec}
  \fi
  \defaultfontfeatures{Ligatures=TeX,Scale=MatchLowercase}
\fi
% use upquote if available, for straight quotes in verbatim environments
\IfFileExists{upquote.sty}{\usepackage{upquote}}{}
% use microtype if available
\IfFileExists{microtype.sty}{%
\usepackage{microtype}
\UseMicrotypeSet[protrusion]{basicmath} % disable protrusion for tt fonts
}{}
\usepackage{hyperref}
\hypersetup{unicode=true,
            pdftitle={A pre-registered multi-site study investigating the effects of inhibitory control training on automatic action tendencies for unhealthy foods},
            pdfauthor={Loukia Tzavella, Ernst-August Doelle, Christopher D. Chambers, Natalia Lawrence, Katherine S. Button, Elizabeth Hart, Natalie Holmes, Kimberley Houghton, Nina Badkar, Ellie Macey, Amy-Jayne Braggins, Felicity Murray, \& Rachel C. Adams},
            pdfkeywords={inhibitory control training, go/no-go, foods, devaluation, action
tendencies, approach bias},
            pdfborder={0 0 0},
            breaklinks=true}
\urlstyle{same}  % don't use monospace font for urls
\usepackage{graphicx,grffile}
\makeatletter
\def\maxwidth{\ifdim\Gin@nat@width>\linewidth\linewidth\else\Gin@nat@width\fi}
\def\maxheight{\ifdim\Gin@nat@height>\textheight\textheight\else\Gin@nat@height\fi}
\makeatother
% Scale images if necessary, so that they will not overflow the page
% margins by default, and it is still possible to overwrite the defaults
% using explicit options in \includegraphics[width, height, ...]{}
\setkeys{Gin}{width=\maxwidth,height=\maxheight,keepaspectratio}
\IfFileExists{parskip.sty}{%
\usepackage{parskip}
}{% else
\setlength{\parindent}{0pt}
\setlength{\parskip}{6pt plus 2pt minus 1pt}
}
\setlength{\emergencystretch}{3em}  % prevent overfull lines
\providecommand{\tightlist}{%
  \setlength{\itemsep}{0pt}\setlength{\parskip}{0pt}}
\setcounter{secnumdepth}{0}
% Redefines (sub)paragraphs to behave more like sections
\ifx\paragraph\undefined\else
\let\oldparagraph\paragraph
\renewcommand{\paragraph}[1]{\oldparagraph{#1}\mbox{}}
\fi
\ifx\subparagraph\undefined\else
\let\oldsubparagraph\subparagraph
\renewcommand{\subparagraph}[1]{\oldsubparagraph{#1}\mbox{}}
\fi

%%% Use protect on footnotes to avoid problems with footnotes in titles
\let\rmarkdownfootnote\footnote%
\def\footnote{\protect\rmarkdownfootnote}


  \title{A pre-registered multi-site study investigating the effects of
inhibitory control training on automatic action tendencies for unhealthy
foods}
    \author{Loukia Tzavella\textsuperscript{1}, Ernst-August
Doelle\textsuperscript{1,2}, Christopher D. Chambers\textsuperscript{1},
Natalia Lawrence\textsuperscript{2}, Katherine S.
Button\textsuperscript{3}, Elizabeth Hart\textsuperscript{4}, Natalie
Holmes\textsuperscript{4}, Kimberley Houghton\textsuperscript{4}, Nina
Badkar\textsuperscript{2}, Ellie Macey\textsuperscript{2}, Amy-Jayne
Braggins\textsuperscript{3}, Felicity Murray\textsuperscript{2}, \&
Rachel C. Adams\textsuperscript{1}}
    \date{}
  
\shorttitle{ICT effects on food action tendencies}
\affiliation{
\vspace{0.5cm}
\textsuperscript{1} Cardiff University Brain Research Imaging Centre, CF24 4HQ, UK\\\textsuperscript{2} School of Psychology, University of Exeter, EX4 4QG, UK\\\textsuperscript{3} Department of Psychology, University of Bath, BS2 7AY, UK\\\textsuperscript{4} School of Psychology, Cardiff University, CF10 3AT, UK}
\keywords{inhibitory control training, go/no-go, foods, devaluation, action tendencies, approach bias\newline\indent Word count: X}
\usepackage{csquotes}
\usepackage{upgreek}
\captionsetup{font=singlespacing,justification=justified}

\usepackage{longtable}
\usepackage{lscape}
\usepackage{multirow}
\usepackage{tabularx}
\usepackage[flushleft]{threeparttable}
\usepackage{threeparttablex}

\newenvironment{lltable}{\begin{landscape}\begin{center}\begin{ThreePartTable}}{\end{ThreePartTable}\end{center}\end{landscape}}

\makeatletter
\newcommand\LastLTentrywidth{1em}
\newlength\longtablewidth
\setlength{\longtablewidth}{1in}
\newcommand{\getlongtablewidth}{\begingroup \ifcsname LT@\roman{LT@tables}\endcsname \global\longtablewidth=0pt \renewcommand{\LT@entry}[2]{\global\advance\longtablewidth by ##2\relax\gdef\LastLTentrywidth{##2}}\@nameuse{LT@\roman{LT@tables}} \fi \endgroup}


\usepackage{lineno}

\linenumbers
\usepackage{xcolor}
\usepackage{multicol}
\usepackage{enumitem}
\usepackage[font=footnotesize,labelfont=bf]{caption}
\usepackage{float}
\usepackage{amsmath}
\usepackage{soul}

\authornote{The research project was conducted as part of the
GW4 Undergraduate Psychology Consortium 2017/2018. This project was
partially supported by the European Research Council (Consolidator
647893; C.D.C.). We gratefully acknowledge Teaching Development Funding,
from the faculty of Humanities and Social Sciences at the University of
Bath for funding travel and room hire costs for the consortium meetings.

Correspondence concerning this article should be addressed to Loukia
Tzavella, Cardiff University Brain Research Imaging Centre, CF24 4HQ,
UK. E-mail:
\href{mailto:tzavellal@cardiff.ac.uk}{\nolinkurl{tzavellal@cardiff.ac.uk}}}



\begin{document}
\maketitle

\section{Introduction}\label{introduction}

\par

There is an increasing interest in the development of behaviour change
interventions for eating behaviours that may arise in an
\enquote{obesogenic environment}, such as overeating. These
interventions largely focus on the cognitive processes that are
responsible for enhancing an individual's self-control, such as response
inhibition. There has been considerable evidence to suggest that such
interventions can result in reduced food consumption in the laboratory
(Allom, Mullan, \& Hagger, 2016 for meta-analyses; see A. Jones et al.,
2016). A common inhibitory control training (ICT) intervention has been
adapted from the go/no-go paradigm, where participants are trained to
inhibit their responses towards highly appetitive foods, and has been
shown to reduce food intake (e.g., Houben \& Jansen, 2015; N. S.
Lawrence, Verbruggen, Morrison, Adams, \& Chambers, 2015a). A potential
mechanism of action behind ICT effects on food consumption is stimulus
devaluation, whereby the evaluations of appetitive foods are reduced
during training to facilitate performance when response inhibition is
required (e.g., Chen, Veling, Dijksterhuis, \& Holland, 2016a). A
possible explanation for this devaluation effect is provided by the
Behaviour Stimulus Interaction (BSI) theory which posits that food
stimuli are devalued when negative affect is induced to reduce the
ongoing conflict between triggered approach reactions to appetitive
foods and the need to inhibit responses towards those stimuli (Chen,
Veling, Dijksterhuis, \& Holland, 2016b; Veling, Holland, \& van
Knippenberg, 2008; Veling, Lawrence, Chen, van Koningsbruggen, \&
Holland, 2017). If the automatic action tendency to approach the food
cue is reduced, the inhibition of responses in the ICT tasks can be
facilitated. In this study we aimed to explore the interaction between
inhibition and approach motivation further in relation to ICT outcomes.
Although the BSI theory focuses on approach tendencies and not
avoidance, we aimed to investigate both automatic action tendencies as
an outcome of go/no-go training in addition to stimulus devaluation.
Specifically, we tested whether go/no-go training changes approach
and/or avoidance tendecies towards unhealthy foods associated with
response inhibition.

\par

In dual-process model frameworks, behaviour is determined by the
interaction of
\texttt{impulsive\textquotesingle{},\ or\ automatic\ and}reflective', or
controlled cognitive processes (Kakoschke, Kemps, \& Tiggemann, 2015;
Strack \& Deutsch, 2004). The reflective system refers to our conscious
and deliberate thoughts that result in reasoned actions which are in
line with our long-term goals. The impulsive system, however, involves
actions that occur without weighting any potential consequences and are
driven by hedonic needs and desires. Eating behaviours that may give
rise to obesity rates, such as overeating, may be explained by a weak
reflective system and/or a strong impulsive system (e.g., N. S.
Lawrence, Hinton, Parkinson, \& Lawrence, 2012; Nederkoorn, Coelho,
Guerrieri, Houben, \& Jansen, 2012).
\hl{I had a perfect reference for this - it's somewhere in my old notes and posters- find it!! and double-check references- also Rachel here had this:for a review see; Stice, Lawrence Kemps, \& Veling, 2016}.
For instance, exposure to unhealthy appetitive food cues might trigger a
conflict between automatic and controlled processing. Attentional (e.g.,
attending to the cue) and motivational (e.g., approaching appetitive
food) processes would be automatic, while choosing an action towards
these foods (e.g., eating vs not eating) while considering the
compatibility of long-term goals (e.g., losing weight and eating
unhealthy foods is not compatible) is a controlled process (Kakoschke et
al., 2015). Indeed, it has been shown that overweight or obese
individuals demonstrate poor self-control and increased impulsivity
across a range of questionnaires and behavioural measures (e.g., Houben,
Nederkoorn, \& Jansen, 2014; Lavagnino, Arnone, Cao, Soares, \&
Selvaraj, 2016; Nederkoorn et al., 2012). Inhibitory control in relation
to unhealthy eating patterns has generally been defined as \enquote{the
ability to inhibit a behavioural impulse in order to attain higher-order
goals, such as weight loss} (Houben, Nederkoorn, \& Jansen, 2012, p.
550). Strengthening the impulsive, or automatic, system may therefore
involve enhancing response inhibition and reducing approach bias towards
appetitive foods.

\par

In a typical ICT paradigm, participants are instructed to make a speeded
choice response to healthy and unhealthy foods, but to withhold that
response when a visual, or auditory, signal is presented.
Signal-stimulus mappings are manipulated so that healthy foods are
associated with a response (\textit{go} foods) and unhealthy foods are
paired with a stop signal (\textit{no-go} foods). In the case of
food-related inhibition training, stopping to unhealthy foods has been
shown to reduce food consumption (Adams, Lawrence, Verbruggen, \&
Chambers, 2017; Houben \& Jansen, 2011, 2015; N. S. Lawrence,
Verbruggen, Morrison, Adams, \& Chambers, 2015b; Veling, Aarts, \&
Papies, 2011), promote healthy food choices (Veling, Aarts, \& Stroebe,
2013; Veling et al.,
2017)\hl{find van Koningsbruggen, Veling, Stroebe, \& Aarts, 2014; and double check 2017 reference}
and has even been associated with increased weight loss (N. S. Lawrence
et al., 2015; Veling, van Koningsbruggen, Aarts, \& Stroebe, 2014).
Several mechanisms have been proposed to explain the effects of
inhibitory control training on behaviour with the most likely method
argued to be stimulus devaluation (Driscoll, Quinn de Launay, \& Fenske,
2018; Veling et al., 2017; but see Jones et al., 2016).

expand on the inhibitory control reflex too.

given the idea proposed by the BSI theory that approach tendencies are
reduced and that is connected to stimulus devaluation and the
theoretical frameworks that suggest an interplay between inhibition and
motivation processes, it shoulld be investigated whether response
inhibition training actually affects implicit approach bias.

\hl{somewhere in here we need to link the literature where AAT is used as a training intervention - check though what were the actual outcomes there was it AAT again - this at least provides evidence that approach tendencies towards foods can be altered in the lab setting..}

\hl{in the discussion we can comment on the importance of methodology for both the AAT and GNG.. e.g. limitations}
- gng shown to be effective when highly appetitive foods are used- check
liking for participants and outline that personalisedsets of stimuli may
be more important - so many approach avoidance taks and we only chose
one variant - and different analyses - eg info from diffusion papers
that we lose information from averaging in this type of tasks

(Chen, Holland, Quandt, Dijksterhuis, \& Veling, 2018)

\section{Methods}\label{methods}

\subsection{Participants}\label{participants}

\par

257 participants were recruited in total from the University campuses of
Cardiff, Bath and Exeter via research participation schemes (e.g.,
Experimental Management system; EMS) and advertisements. Participants
recruited through participation schemes received course credits, whereas
other individuals were offered entry into a prize draw for one of three
£20 shopping vouchers. Participants were informed about the study
eligibility criteria and in order to ensure compliance they completed a
screening survey in the beginning of the study and provided their
consent. They were asked to refrain from eating for 3 hours before the
study and data collection was thus conducted only after
midday.\hl{actually check in the data files} Participants had to be at
least 18 years of age, be fluent in spoken and written English and have
normal or corrected-to-normal vision, including normal colour vision.
Participants were excluded if they were dieting at the time of the
study, with a weight goal and time-frame in mind, had a current and/or
past diagnosis of any eating disorder(s) and had a body-mass-index (BMI)
lower than 18.5 kg/m2 (i.e., underweight category). The study was
approved by the Ethics Committees of Cardiff University, University of
Bath and the University of Exeter.
\hl{need to add info for recruitment from different uni sites}

\subsection{Sampling plan}\label{sampling}

\par

The required sample size was estimated based on a frequentist power
analysis conducted for the primary outcome measure (i.e., change in
approach-avoidance bias, from pre-to post-training, between go and no-go
foods; H1a and H1b) and the stimulus devaluation manipulation check
(i.e., change in food liking, from pre-to-post training, between go and
no-go foods; H2). Both of these effect sizes were in the medium range,
we therefore based our calculations on the primary outcome measure. For
an expected effect size we considered other studies that have measured
approach bias pre-and post-approach-avoidance training (D. Becker,
Jostmann, Wiers, \& Holland, 2015; Schumacher, Kemps, \& Tiggemann,
2016). Both studies reported an effect size of
\textit{$\eta$η\textsubscript{p}\textsuperscript{2}}=0.07 which
corresponds to a \enquote{medium} effect size. D. Becker et al. (2015)
\hl{double check it's the same paper as I had 2014 here} also reported
two non-significant results, although effect sizes were not
provided\textbackslash{}footnote\{Note, however, that D. Becker et al.
(2015) compared an active group with 90:10 mapping (i.e.~avoidance of
90\% for unhealthy trials and 10\% healthy trials) to a control group
with 50:50 mapping whereas Schumacher et al. (2016) compared a 90:10
active group with a 10:90 control group.\}. We therefore took a
conservative approach when calculating our sample size. Firstly, we
reduced the effect size by 33\% (i.e., \textit{dz} = 0.34) to account
for publication bias (Button et al., 2013) and secondly we used an alpha
of 0.005, which has recently been recommended for any research that
cannot be considered a direct replication and can increase the
reliability of new discoveries (Benjamin et al., 2018). Based on a
priori power calculations using G*Power (Faul, Erdfelder, Buchner, \&
Lang, 2009) we estimated that a total sample of 149 participants was
necessary for 90\% power.

\par  

The sampling method and power analysis of the study followed a
conservative frequentist approach, but the pre-registered analyses were
based on a Bayesian framework (see \hl{Pre-registered analyses}).
Frequentist analyses were also reported in a supplementary fashion.
Bayes factors (BFs) informed the interpretations of the results and
although debate exists about labelling evidence in terms of BFs (Morey,
2015), we followed the guidelines by (Lee \& Wagenmakers, 2013). A
threshold of BF\textsubscript{10} \textgreater{} 6 was used to indicate
moderate evidence for the alternative hypothesis relative to the null,
and BF\textsubscript{10} \textless{} 1/6 reflected moderate evidence for
the null relative to the respective alternative hypothesis. Bayes factor
analyses were favoured for drawing conclusions from the study, as they
would allow us to interpret null outcomes as evidence of absence when
traditional analyses would not make such inferences feasible. For
frequentists analyses, an alpha level of 0.005 was used.

\subsection{Procedure}\label{procedure}

\hl{TO BE COMPLETED- NICE FIGURE IN PROGRESS :)}

\subsection{Go/No-go training}\label{gng}

\par

The Go/No-Go (GNG) training paradigm involved \enquote{go} and
\enquote{no-go} responses to six pre-selected appetitive food
categories. Food categories differed in terms of taste, so that three
foods were savoury (i.e., pizza, crisps, chips) and three foods were
sweet (i.e., biscuits, chocolate, cake). Two food categories were
randomly assigned to each training condition (go, no-go, filler foods)
in the beginning of the experiment and food taste was counterbalanced so
that each condition had one sweet and one savoury food. There were three
training conditions according to the mapping of foods to signal (no-go)
and no-signal (go) trials in the GNG paradigm. All go foods appeared in
go trials and all no-go foods were presented with the signal (see Figure
1, panel C). Control foods appeared on both go and no-go trials with
equal probability (i.e., 50\% signal and 50\% no-signal trial mapping).
Each food category had three exemplars which appeared twice in each
block.

\par

All foods were presented on either the left or right hand side of the
screen within a rectangle for 1250ms (see Figure 1, panel B).
Participants were asked to respond to the location of the food as
quickly and as accurately as possible by pressing the \enquote{C} and
\enquote{M} buttons on the keyboard with their left and right index
fingers, respectively. The central rectangle remained on the screen
throughout the training, including the inter-trial-interval (ITI), which
was 1250ms. On signal trials, the rectangle turned \enquote{bold},
indicating that participants should withhold their response. In line
with the GNG training paradigm, this signal appeared on stimulus onset
(i.e., no delay between stimulus and signal) and stayed on the screen
until the end of the trial. A correct response on no-signal trials was
registered when participants responded accurately to the location of the
food within the time limit and a successful stop (i.e., correct signal
trial) was considered when participants did not respond during the trial
time window at all. Incorrect responses in no-signal trials refer to
either to a wrong location judgment or a missed response. Left and right
responses were counterbalanced across all manipulated variables for each
type of trial. Training was split into 6 blocks of 36 trials (i.e., 216
trials in total) and lasted approximately 10 minutes with inter-block
breaks (15 seconds). Task practice included 12 trials of go and no-go
responses (50\%-50\%) and participants responded to the location of grey
squares, instead of food pictures. Feedback was presented during the ITI
for practice trials only (i.e, \enquote{CORRECT} or \enquote{INCORRECT}
in green and red text, respectively).

\subsection{Approach avoidance task}\label{aat}

\par

The approach-avoidance task (AAT) was adapted from an existent paradigm
(M. Rinck \& Becker, 2007; R. W. Wiers, Rinck, Dictus, \& Van Den
Wildenberg, 2009), which involves \enquote{pull} (i.e., towards self)
and \enquote{push} (i.e, away from self) movements of a joystick. Each
type of motor response is paired with visual feedback so that when the
joystick is pulled, the image gets bigger (zoom-in) and when it is
pushed, the image gets smaller (zoom-out). This \enquote{zooming}
effects acts as an exteroceptive cue of either an approach or avoidance
response (Neumann \& Strack, 2000). This feature of the joystick AAT
complements the proprioceptive properties of the task, where responses
requiring arm flexion and extension correspond to approach and avoidance
trials, respectively. This task also disambiguates approach and
avoidance responses by using the \enquote{zooming} feature (R. W. Wiers
et al., 2009). For example, arm extension could indicate an approach
response towards an appetitive food (object-reference) or an avoidance
response where the food is pushed away from the body/self
(self-reference; Phaf, Mohr, Rotteveel, \& Wicherts, 2014). The visual
feedback thus provides the self-reference attribute to the responses
(e.g., object comes closer to one's body). We also adopted the
evaluation-irrelevant feature of the paradigm, whereby participants
respond according to the format of (portrait or landscape; e.g., R. W.
Wiers, Rinck, Kordts, Houben, \& Strack, 2010).

\par

AAT responses involved \enquote{push} and \enquote{pull} movements of
the computer mouse. Food stimuli were presented in the centre of the
screen and participants were instructed to pull the mouse towards them
or push the mouse away from them according to whether the image was in
portrait or landscape format (see Figure \hl{X}). Response-format
assignments were approximately counterbalanced \hl{check in data} across
participants. Instructions highlighted moving the mouse cursor until it
reaches the end of the screen (top or bottom edge) for a correct
response to be registered and making smooth whole-arm movements.
Participants had 1500ms to respond after the stimulus appeared. Each
trial started with a central \enquote{X} on the screen and participants
had to click on it to begin. The ITI was 500 ms and there was no delay
between the \enquote{X} click response and the stimulus onset. In order
to account for the natural movement of the mouse, pixel tolerance
\hl{how many pixels +/- X px
} was added to every mouse movement, including movement initiation in
the beginning of the trial. {[}Add what a correct response was{]}

\section{Results}\label{results}

\section{Discussion}\label{discussion}

\newpage

\section{References}\label{references}

\begingroup
\setlength{\parindent}{-0.5in} \setlength{\leftskip}{0.5in}

\hypertarget{refs}{}
\hypertarget{ref-adams_training_2017}{}
Adams, R. C., Lawrence, N. S., Verbruggen, F., \& Chambers, C. D.
(2017). Training response inhibition to reduce food consumption:
Mechanisms, stimulus specificity and appropriate training protocols.
\emph{Appetite}, \emph{109}, 11--23.
\url{https://doi.org/10.1016/j.appet.2016.11.014}

\hypertarget{ref-allom_does_2016}{}
Allom, V., Mullan, B., \& Hagger, M. (2016). Does inhibitory control
training improve health behaviour? A meta-analysis. \emph{Health
Psychol. Rev.}, \emph{10}(2), 168--186.
\url{https://doi.org/10.1080/17437199.2015.1051078}

\hypertarget{ref-becker_approach_2015-1}{}
Becker, D., Jostmann, N. B., Wiers, R. W., \& Holland, R. W. (2015).
Approach avoidance training in the eating domain: Testing the
effectiveness across three single session studies. \emph{Appetite},
\emph{85}(June 2015), 58--65.
\url{https://doi.org/10.1016/j.appet.2014.11.017}

\hypertarget{ref-benjamin_redefine_2017}{}
Benjamin, D. J., Berger, J. O., Johannesson, M., Nosek, B. A.,
Wagenmakers, E.-J., Berk, R., \ldots{} Johnson, V. E. (2018). Redefine
statistical significance. \emph{Nature Human Behaviour}, \emph{2},
6--10. \url{https://doi.org/10.1038/s41562-017-0189-z}

\hypertarget{ref-button_power_2013}{}
Button, K. S., Ioannidis, J. P. A., Mokrysz, C., Nosek, B. A., Flint,
J., Robinson, E. S. J., \& Munaf\a`o, M. R. (2013). Power failure: Why
small sample size undermines the reliability of neuroscience. \emph{Nat.
Rev. Neurosci.}, \emph{14}(5), 365--376.
\url{https://doi.org/10.1038/nrn3475}

\hypertarget{ref-chen_when_2018}{}
Chen, Z., Holland, R., Quandt, J., Dijksterhuis, A., \& Veling, H.
(2018). When mere action versus inaction leads to robust preference
change. \url{https://doi.org/10.17605/OSF.IO/ZY9W3}

\hypertarget{ref-chen_how_2016-1}{}
Chen, Z., Veling, H., Dijksterhuis, A., \& Holland, R. W. (2016a). How
does not responding to appetitive stimuli cause devaluation: Evaluative
conditioning or response inhibition? \emph{Journal of Experimental
Psychology: General}, \emph{145}(12), 1687--1701.
\url{https://doi.org/10.1037/xge0000236}

\hypertarget{ref-chen_how_2016}{}
Chen, Z., Veling, H., Dijksterhuis, A., \& Holland, R. W. (2016b). How
does not responding to appetitive stimuli cause devaluation: Evaluative
conditioning or response inhibition? \emph{Journal of Experimental
Psychology: General}, \emph{145}(12), 1687--1701.
\url{https://doi.org/10.1037/xge0000236}

\hypertarget{ref-faul_statistical_2009}{}
Faul, F., Erdfelder, E., Buchner, A., \& Lang, A.-G. (2009). Statistical
power analyses using G*Power 3.1: Tests for correlation and regression
analyses. \emph{Behav. Res. Methods}, \emph{41}(4), 1149--1160.
\url{https://doi.org/10.3758/BRM.41.4.1149}

\hypertarget{ref-houben_training_2011}{}
Houben, K., \& Jansen, A. (2011). Training inhibitory control. A recipe
for resisting sweet temptations. \emph{Appetite}, \emph{56}(2),
345--349. \url{https://doi.org/10.1016/j.appet.2010.12.017}

\hypertarget{ref-houben_chocolate_2015}{}
Houben, K., \& Jansen, A. (2015). Chocolate equals stop:
Chocolate-specific inhibition training reduces chocolate intake and go
associations with chocolate. \emph{Appetite}, \emph{87}, 318--323.
\url{https://doi.org/10.1016/j.appet.2015.01.005}

\hypertarget{ref-houben_too_2012}{}
Houben, K., Nederkoorn, C., \& Jansen, A. (2012). Too tempting to
resist? Past success at weight control rather than dietary restraint
determines exposure-induced disinhibited eating. \emph{Appetite},
\emph{59}(2), 550--555.
\url{https://doi.org/10.1016/j.appet.2012.07.004}

\hypertarget{ref-houben_eating_2014}{}
Houben, K., Nederkoorn, C., \& Jansen, A. (2014). Eating on impulse: The
relation between overweight and food-specific inhibitory control.
\emph{Obesity}, \emph{22}(5), 2013--2015.
\url{https://doi.org/10.1002/oby.20670}

\hypertarget{ref-jones_inhibitory_2016}{}
Jones, A., Di Lemma, L. C., Robinson, E., Christiansen, P., Nolan, S.,
Tudur-Smith, C., \& Field, M. (2016). Inhibitory control training for
appetitive behaviour change: A meta-analytic investigation of mechanisms
of action and moderators of effectiveness. \emph{Appetite}, \emph{97},
16--28. \url{https://doi.org/10.1016/j.appet.2015.11.013}

\hypertarget{ref-kakoschke_combined_2015-1}{}
Kakoschke, N., Kemps, E., \& Tiggemann, M. (2015). Combined effects of
cognitive bias for food cues and poor inhibitory control on unhealthy
food intake. \emph{Appetite}, \emph{87}, 358--364.
\url{https://doi.org/10.1016/j.appet.2015.01.004}

\hypertarget{ref-lavagnino_inhibitory_2016}{}
Lavagnino, L., Arnone, D., Cao, B., Soares, J. C., \& Selvaraj, S.
(2016). Inhibitory control in obesity and binge eating disorder: A
systematic review and meta-analysis of neurocognitive and neuroimaging
studies. \emph{Neurosci. Biobehav. Rev.}, \emph{68}, 714--726.
\url{https://doi.org/10.1016/j.neubiorev.2016.06.041}

\hypertarget{ref-lawrence_nucleus_2012-1}{}
Lawrence, N. S., Hinton, E. C., Parkinson, J. A., \& Lawrence, A. D.
(2012). Nucleus accumbens response to food cues predicts subsequent
snack consumption in women and increased body mass index in those with
reduced self-control. \emph{NeuroImage}, \emph{63}(1), 415--422.
\url{https://doi.org/10.1016/j.neuroimage.2012.06.070}

\hypertarget{ref-lawrence_training_2015}{}
Lawrence, N. S., O'Sullivan, J., Parslow, D., Javaid, M., Adams, R. C.,
Chambers, C. D., \ldots{} Verbruggen, F. (2015). Training response
inhibition to food is associated with weight loss and reduced energy
intake. \emph{Appetite}, \emph{95}, 17--28.
\url{https://doi.org/10.1016/j.appet.2015.06.009}

\hypertarget{ref-lawrence_stopping_2015-3}{}
Lawrence, N. S., Verbruggen, F., Morrison, S., Adams, R. C., \&
Chambers, C. D. (2015a). Stopping to food can reduce intake. Effects of
stimulus-specificity and individual differences in dietary restraint.
\emph{Appetite}, \emph{85}, 91--103.
\url{https://doi.org/10.1016/j.appet.2014.11.006}

\hypertarget{ref-lawrence_stopping_2015}{}
Lawrence, N. S., Verbruggen, F., Morrison, S., Adams, R. C., \&
Chambers, C. D. (2015b). Stopping to food can reduce intake. Effects of
stimulus-specificity and individual differences in dietary restraint.
\emph{Appetite}, \emph{85}, 91--103.
\url{https://doi.org/10.1016/j.appet.2014.11.006}

\hypertarget{ref-lee_bayesian_2013}{}
Lee, M. D., \& Wagenmakers, E.-J. (2013). \emph{Bayesian Cognitive
Modeling: A Practical Course}. Cambridge University Press.
\url{https://doi.org/10.1017/CBO9781139087759}

\hypertarget{ref-richard_d._morey_verbal_2015}{}
Morey, R. D. (2015). On verbal categories for the interpretation of
Bayes factors.

\hypertarget{ref-nederkoorn_specificity_2012}{}
Nederkoorn, C., Coelho, J. S., Guerrieri, R., Houben, K., \& Jansen, A.
(2012). Specificity of the failure to inhibit responses in overweight
children. \emph{Appetite}, \emph{59}(2), 409--413.
\url{https://doi.org/10.1016/j.appet.2012.05.028}

\hypertarget{ref-neumann_approach_2000}{}
Neumann, R., \& Strack, F. (2000). Approach and Avoidance: The Influence
of Proprioceptive and Exteroceptive Cues on Encoding of Affective
Information. \emph{J. Personal. Soc. Psychol.}, \emph{79}(1), 39--48.
\url{https://doi.org/10.1037//0022-3514.79.1.39}

\hypertarget{ref-phaf_approach_2014}{}
Phaf, R. H., Mohr, S. E., Rotteveel, M., \& Wicherts, J. M. (2014).
Approach, avoidance, and affect: A meta-analysis of approach-avoidance
tendencies in manual reaction time tasks. \emph{Front. Psychol.},
\emph{5}(378), 1--16. \url{https://doi.org/10.3389/fpsyg.2014.00378}

\hypertarget{ref-rinck_approach_2007}{}
Rinck, M., \& Becker, E. S. (2007). Approach and avoidance in fear of
spiders. \emph{Journal of Behavior Therapy and Experimental Psychiatry},
\emph{38}(2), 105--120.
\url{https://doi.org/10.1016/j.jbtep.2006.10.001}

\hypertarget{ref-schumacher_bias_2016}{}
Schumacher, S. E., Kemps, E., \& Tiggemann, M. (2016). Bias modification
training can alter approach bias and chocolate consumption.
\emph{Appetite}, \emph{96}, 219--224.
\url{https://doi.org/10.1016/j.appet.2015.09.014}

\hypertarget{ref-strack_reflective_2004}{}
Strack, F., \& Deutsch, R. (2004). Reflective and Impulsive Determinants
of Social Behavior. \emph{Personality and Social Psychology Review},
\emph{8}(3), 28.

\hypertarget{ref-veling_using_2011}{}
Veling, H., Aarts, H., \& Papies, E. K. (2011). Using stop signals to
inhibit chronic dieters' responses toward palatable foods. \emph{Behav.
Res. Ther.}, \emph{49}(11), 771--780.
\url{https://doi.org/10.1016/j.brat.2011.08.005}

\hypertarget{ref-veling_stop_2013}{}
Veling, H., Aarts, H., \& Stroebe, W. (2013). Stop signals decrease
choices for palatable foods through decreased food evaluation.
\emph{Front. Psychol.}, \emph{4}(875), 1--7.
\url{https://doi.org/10.3389/fpsyg.2013.00875}

\hypertarget{ref-veling_training_2017-1}{}
Veling, H., Chen, Z., Tombrock, M. C., M. Verpaalen, I. a., Schmitz, L.
I., Dijksterhuis, A., \& Holland, R. W. (2017). Training Impulsive
Choices for Healthy and Sustainable Food. \emph{J. Exp. Psychol. Appl.},
\emph{23}(1), 1--14. \url{https://doi.org/10.1037/xap0000112}

\hypertarget{ref-veling_when_2008}{}
Veling, H., Holland, R. W., \& van Knippenberg, A. (2008). When approach
motivation and behavioral inhibition collide: Behavior regulation
through stimulus devaluation. \emph{Journal of Experimental Social
Psychology}, \emph{44}(4), 1013--1019.
\url{https://doi.org/10.1016/j.jesp.2008.03.004}

\hypertarget{ref-veling_what_2017}{}
Veling, H., Lawrence, N. S., Chen, Z., van Koningsbruggen, G. M., \&
Holland, R. W. (2017). What Is Trained During Food Go/No-Go Training? A
Review Focusing on Mechanisms and a Research Agenda. \emph{Curr. Addict.
Reports}, \emph{4}(1), 35--41.
\url{https://doi.org/10.1007/s40429-017-0131-5}

\hypertarget{ref-veling_targeting_2014}{}
Veling, H., van Koningsbruggen, G. M., Aarts, H., \& Stroebe, W. (2014).
Targeting impulsive processes of eating behavior via the internet.
Effects on body weight. \emph{Appetite}, \emph{78}, 102--109.
\url{https://doi.org/10.1016/j.appet.2014.03.014}

\hypertarget{ref-wiers_relatively_2009}{}
Wiers, R. W., Rinck, M., Dictus, M., \& Van Den Wildenberg, E. (2009).
Relatively strong automatic appetitive action-tendencies in male
carriers of the OPRM1 G-allele. \emph{Genes, Brain Behav.}, \emph{8}(1),
101--106. \url{https://doi.org/10.1111/j.1601-183X.2008.00454.x}

\hypertarget{ref-wiers_retraining_2010}{}
Wiers, R. W., Rinck, M., Kordts, R., Houben, K., \& Strack, F. (2010).
Retraining automatic action-tendencies to approach alcohol in hazardous
drinkers. \emph{Addiction}, \emph{105}(2), 279--287.
\url{https://doi.org/10.1111/j.1360-0443.2009.02775.x}

\endgroup --\textgreater{}


\end{document}
