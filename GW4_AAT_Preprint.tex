\documentclass[man,floatsintext]{apa6}
\usepackage{lmodern}
\usepackage{amssymb,amsmath}
\usepackage{ifxetex,ifluatex}
\usepackage{fixltx2e} % provides \textsubscript
\ifnum 0\ifxetex 1\fi\ifluatex 1\fi=0 % if pdftex
  \usepackage[T1]{fontenc}
  \usepackage[utf8]{inputenc}
\else % if luatex or xelatex
  \ifxetex
    \usepackage{mathspec}
  \else
    \usepackage{fontspec}
  \fi
  \defaultfontfeatures{Ligatures=TeX,Scale=MatchLowercase}
\fi
% use upquote if available, for straight quotes in verbatim environments
\IfFileExists{upquote.sty}{\usepackage{upquote}}{}
% use microtype if available
\IfFileExists{microtype.sty}{%
\usepackage{microtype}
\UseMicrotypeSet[protrusion]{basicmath} % disable protrusion for tt fonts
}{}
\usepackage{hyperref}
\hypersetup{unicode=true,
            pdftitle={A multi-site investigating the effects of inhibitory control training on automatic action tendencies for unhealthy foods},
            pdfauthor={Loukia Tzavella, Ernst-August Doelle, Christopher D. Chambers, Natalia Lawrence, Katherine S. Button, Elizabeth Hart, Natalie Holmes, Kimberley Houghton, Nina Badkar, Ellie Macey, Amy-Jayne Braggins, Felicity Murray, \& Rachel C. Adams},
            pdfkeywords={inhibitory control training, go/no-go, foods, devaluation, action
tendencies, approach bias},
            pdfborder={0 0 0},
            breaklinks=true}
\urlstyle{same}  % don't use monospace font for urls
\usepackage{graphicx,grffile}
\makeatletter
\def\maxwidth{\ifdim\Gin@nat@width>\linewidth\linewidth\else\Gin@nat@width\fi}
\def\maxheight{\ifdim\Gin@nat@height>\textheight\textheight\else\Gin@nat@height\fi}
\makeatother
% Scale images if necessary, so that they will not overflow the page
% margins by default, and it is still possible to overwrite the defaults
% using explicit options in \includegraphics[width, height, ...]{}
\setkeys{Gin}{width=\maxwidth,height=\maxheight,keepaspectratio}
\IfFileExists{parskip.sty}{%
\usepackage{parskip}
}{% else
\setlength{\parindent}{0pt}
\setlength{\parskip}{6pt plus 2pt minus 1pt}
}
\setlength{\emergencystretch}{3em}  % prevent overfull lines
\providecommand{\tightlist}{%
  \setlength{\itemsep}{0pt}\setlength{\parskip}{0pt}}
\setcounter{secnumdepth}{0}
% Redefines (sub)paragraphs to behave more like sections
\ifx\paragraph\undefined\else
\let\oldparagraph\paragraph
\renewcommand{\paragraph}[1]{\oldparagraph{#1}\mbox{}}
\fi
\ifx\subparagraph\undefined\else
\let\oldsubparagraph\subparagraph
\renewcommand{\subparagraph}[1]{\oldsubparagraph{#1}\mbox{}}
\fi

%%% Use protect on footnotes to avoid problems with footnotes in titles
\let\rmarkdownfootnote\footnote%
\def\footnote{\protect\rmarkdownfootnote}


  \title{A multi-site investigating the effects of inhibitory control training on
automatic action tendencies for unhealthy foods}
    \author{Loukia Tzavella\textsuperscript{1}, Ernst-August
Doelle\textsuperscript{1,2}, Christopher D. Chambers\textsuperscript{1},
Natalia Lawrence\textsuperscript{2}, Katherine S.
Button\textsuperscript{3}, Elizabeth Hart\textsuperscript{4}, Natalie
Holmes\textsuperscript{4}, Kimberley Houghton\textsuperscript{4}, Nina
Badkar\textsuperscript{2}, Ellie Macey\textsuperscript{2}, Amy-Jayne
Braggins\textsuperscript{3}, Felicity Murray\textsuperscript{2}, \&
Rachel C. Adams\textsuperscript{1}}
    \date{}
  
\shorttitle{ICT effects on food action tendencies}
\affiliation{
\vspace{0.5cm}
\textsuperscript{1} Cardiff University Brain Research Imaging Centre, CF24 4HQ, UK\\\textsuperscript{2} School of Psychology, University of Exeter, EX4 4QG, UK\\\textsuperscript{3} Department of Psychology, University of Bath, BS2 7AY, UK\\\textsuperscript{4} School of Psychology, Cardiff University, CF10 3AT, UK}
\keywords{inhibitory control training, go/no-go, foods, devaluation, action tendencies, approach bias\newline\indent Word count: X}
\usepackage{csquotes}
\usepackage{upgreek}
\captionsetup{font=singlespacing,justification=justified}

\usepackage{longtable}
\usepackage{lscape}
\usepackage{multirow}
\usepackage{tabularx}
\usepackage[flushleft]{threeparttable}
\usepackage{threeparttablex}

\newenvironment{lltable}{\begin{landscape}\begin{center}\begin{ThreePartTable}}{\end{ThreePartTable}\end{center}\end{landscape}}

\makeatletter
\newcommand\LastLTentrywidth{1em}
\newlength\longtablewidth
\setlength{\longtablewidth}{1in}
\newcommand{\getlongtablewidth}{\begingroup \ifcsname LT@\roman{LT@tables}\endcsname \global\longtablewidth=0pt \renewcommand{\LT@entry}[2]{\global\advance\longtablewidth by ##2\relax\gdef\LastLTentrywidth{##2}}\@nameuse{LT@\roman{LT@tables}} \fi \endgroup}


\usepackage{lineno}

\linenumbers
\usepackage{xcolor}
\usepackage{multicol}
\usepackage{enumitem} \setlist{nosep}
\usepackage[font=footnotesize,labelfont=bf]{caption}
\usepackage{float}
\usepackage{amsmath}
\usepackage{soul}
\usepackage{caption}
\usepackage{booktabs}

\authornote{The research project was conducted as part of the
GW4 Undergraduate Psychology Consortium 2017/2018 and was partially
supported by the European Research Council (Consolidator 647893;
C.D.C.). We also gratefully acknowledge Teaching Development Funding,
from the faculty of Humanities and Social Sciences at the University of
Bath for funding travel and room hire costs for the consortium meetings.

Correspondence concerning this article should be addressed to Loukia
Tzavella, Cardiff University Brain Research Imaging Centre, CF24 4HQ,
UK. E-mail:
\href{mailto:tzavellal@cardiff.ac.uk}{\nolinkurl{tzavellal@cardiff.ac.uk}}}



\begin{document}
\maketitle

\section{Introduction}\label{introduction}

\par

There is an increasing interest in the development of behaviour change
interventions for eating behaviours that may arise in an
\enquote{obesogenic environment}, such as the over-consumption of
high-calorie foods. These interventions largely focus on the cognitive
processes that are responsible for enhancing an individual's
self-control, such as response inhibition. There has been considerable
evidence to suggest that such interventions can result in reduced food
consumption in the laboratory (Allom, Mullan, \& Hagger, 2016 for
meta-analyses; see Jones et al., 2016). A common inhibitory control
training (ICT) intervention has been adapted from the go/no-go paradigm,
where participants are trained to inhibit their responses towards highly
appetitive foods, and has been shown to reduce food intake (e.g., Houben
\& Jansen, 2015; N. S. Lawrence, Verbruggen, Morrison, Adams, \&
Chambers, 2015a). A potential mechanism of action behind ICT effects on
food consumption is stimulus devaluation, whereby the evaluations of
appetitive foods are reduced during training to facilitate performance
when response inhibition is required (e.g., Chen, Veling, Dijksterhuis,
\& Holland, 2016a). A possible explanation for this devaluation effect
is provided by the Behaviour Stimulus Interaction (BSI) theory which
posits that food stimuli are devalued when negative affect is induced to
reduce the ongoing conflict between triggered approach reactions to
appetitive foods and the need to inhibit responses towards those stimuli
(Chen, Veling, Dijksterhuis, \& Holland, 2016b; Veling, Holland, \& van
Knippenberg, 2008; Veling, Lawrence, Chen, van Koningsbruggen, \&
Holland, 2017). If the automatic action tendency to approach the food
cue is reduced, the inhibition of responses in the ICT tasks can be
facilitated. In this study we aimed to explore the interaction between
inhibition and approach motivation further in relation to ICT outcomes.
Although the BSI theory focuses on approach tendencies and not
avoidance, we aimed to investigate both automatic action tendencies as
an outcome of go/no-go training in addition to stimulus devaluation.
Specifically, we tested whether go/no-go training changes approach
and/or avoidance tendecies towards unhealthy foods associated with
response inhibition.

\par

In dual-process model frameworks, behaviour is determined by the
interaction of
\texttt{impulsive\textquotesingle{},\ or\ automatic\ and}reflective', or
controlled cognitive processes (Kakoschke, Kemps, \& Tiggemann, 2015;
Strack \& Deutsch, 2004). The reflective system refers to our conscious
and deliberate thoughts that result in reasoned actions which are in
line with our long-term goals. The impulsive system, however, involves
actions that occur without weighting any potential consequences and are
driven by hedonic needs and desires. Eating behaviours that may give
rise to obesity rates, such as overeating, may be explained by a weak
reflective system and/or a strong impulsive system (e.g., N. S.
Lawrence, Hinton, Parkinson, \& Lawrence, 2012; Nederkoorn, Coelho,
Guerrieri, Houben, \& Jansen, 2012).
\hl{I had a perfect reference for this - it's somewhere in my old notes and posters- find it!! and double-check references- also Rachel here had this:for a review see; Stice, Lawrence Kemps, \& Veling, 2016}.
For instance, exposure to unhealthy appetitive food cues might trigger a
conflict between automatic and controlled processing. Attentional (e.g.,
attending to the cue) and motivational (e.g., approaching appetitive
food) processes would be automatic, while choosing an action towards
these foods (e.g., eating vs not eating) while considering the
compatibility of long-term goals (e.g., losing weight and eating
unhealthy foods is not compatible) is a controlled process (Kakoschke et
al., 2015). Indeed, it has been shown that overweight or obese
individuals demonstrate poor self-control and increased impulsivity
across a range of questionnaires and behavioural measures (e.g., Houben,
Nederkoorn, \& Jansen, 2014; Lavagnino, Arnone, Cao, Soares, \&
Selvaraj, 2016; Nederkoorn et al., 2012). Inhibitory control in relation
to unhealthy eating patterns has generally been defined as \enquote{the
ability to inhibit a behavioural impulse in order to attain higher-order
goals, such as weight loss} (Houben, Nederkoorn, \& Jansen, 2012, p.
550). Strengthening the impulsive, or automatic, system may therefore
involve enhancing response inhibition and reducing approach bias towards
appetitive foods.

\par

In a typical ICT paradigm, participants are instructed to make a speeded
choice response to healthy and unhealthy foods, but to withhold that
response when a visual, or auditory, signal is presented.
Signal-stimulus mappings are manipulated so that healthy foods are
associated with a response (\textit{go} foods) and unhealthy foods are
paired with a stop signal (\textit{no-go} foods). In the case of
food-related inhibition training, stopping to unhealthy foods has been
shown to reduce food consumption (Adams, Lawrence, Verbruggen, \&
Chambers, 2017; Houben \& Jansen, 2011, 2015; N. S. Lawrence,
Verbruggen, Morrison, Adams, \& Chambers, 2015b; Veling, Aarts, \&
Papies, 2011), promote healthy food choices (Veling, Aarts, \& Stroebe,
2013; Veling et al.,
2017)\hl{find van Koningsbruggen, Veling, Stroebe, \& Aarts, 2014; and double check 2017 reference}
and has even been associated with increased weight loss (N. S. Lawrence
et al., 2015a; Veling, van Koningsbruggen, Aarts, \& Stroebe, 2014).
Several mechanisms have been proposed to explain the effects of
inhibitory control training on behaviour with the most likely method
argued to be stimulus devaluation (Driscoll, Quinn de Launay, \& Fenske,
2018; Veling et al., 2017; but see Jones et al. (2016)).

expand on the inhibitory control reflex too.

given the idea proposed by the BSI theory that approach tendencies are
reduced and that is connected to stimulus devaluation and the
theoretical frameworks that suggest an interplay between inhibition and
motivation processes, it shoulld be investigated whether response
inhibition training actually affects implicit approach bias.

\hl{somewhere in here we need to link the literature where AAT is used as a training intervention - check though what were the actual outcomes there was it AAT again - this at least provides evidence that approach tendencies towards foods can be altered in the lab setting..}

\hl{in the discussion we can comment on the importance of methodology for both the AAT and GNG.. e.g. limitations}
- gng shown to be effective when highly appetitive foods are used- check
liking for participants and outline that personalisedsets of stimuli may
be more important - so many approach avoidance taks and we only chose
one variant - and different analyses - eg info from diffusion papers
that we lose information from averaging in this type of tasks

(Chen, Holland, Quandt, Dijksterhuis, \& Veling, 2018)

\section{Hypotheses}\label{hypotheses}

All hypotheses described in this section are confirmatory and have been
pre-registered\footnote{Exact hypotheses from the pre-registered protocol have been re-ordered according to outcomes for clarity. We report no deviations from the protocol for the hypotheses and corresponding statistical tests.}
on the Open Science Framework (\url{https://osf.io/wav8p/}). We examined
the effects of ICT (go/no-go training; see \textit{\nameref{gng}}) on
automatic action tendencies (see \textit{\nameref{aat}}) and liking (see
\textit{\nameref{food_ratings}}) for unhealthy foods. These effects were
investigated using change scores from pre-to post-training for both
outcomes (H1, H3). Training condition was also expected to have an
effect on food choice behaviour (H2; see
\textit{\nameref{food_choice}}). The study assessed contingency learning
mechanisms for the training paradigm, as a manipulation check (H4).

\subsection{Training effects on automatic action
tendencies}\label{training-effects-on-automatic-action-tendencies}

The primary outcome measure in the study was the change in automatic
action tendencies from pre-to post- ICT training for the foods
associated with different conditions (go, no-go and control - see Figure
\ref{fig:procedure}). Action tendencies were indirectly measured via the
AAT and approach-avoidance bias scores were obtained by substracting the
median response times (RTs) in avoid trials (push action) from the RTs
in approach trials (pull action) at the participant level, for each
training condition and then calculating the change from pre-to
post-training. Based on reported effects of AAT training on food
consumption (\hl{check outcomes for sign studies and add citation}) and
the potential interaction between motivation, affect and training
mechanisms, we hypothesized that ICT training would lead to an increase
in approach bias for go goods and reduction in approach bias for no-go
goods, compared to the control foods.

\noindent H1. There will be moderate evidence for an effect of training
condition (go, no-go, control) on the change in approach-avoid bias
scores from pre-to post-training.

\begin{itemize}
\item[H1a.] Participants will show a reduction in approach bias for no-go foods compared to the control foods, from pre-to post- training.
\item[H1b.] Participants will have increased approach bias towards go foods relative to the control foods, from pre-to post- training.
\end{itemize}

\subsection{Training effects on impulsive food
choices}\label{training-effects-on-impulsive-food-choices}

As a secondary outcome, we also examined whether ICT would affect
impulsive food choices for unhealthy foods. We compared the
probabilities of choosing foods from each training condition and in line
with previous findings
(\hl{when you work on the intro, keep a record of which studies found effects on impulsive food choice and what comparison they had - e.g. nogo vs go?}),
we expected that \hl{add here}

\noindent H2. Two Bayesian paired samples t-tests were conducted for the
mean proportions of selected foods in the go and no-go training
condition compared to the control.

\begin{itemize}
\item[H2a.] Participants will show reduced choices for no-go foods relative to the control foods.
\item[H2b.] Participants will show increased choices for go foods relative to the control foods.
\end{itemize}

\subsection{Manipulation check 1: Stimulus
devaluation}\label{manipulation-check-1-stimulus-devaluation}

The mean change in food liking ratings from pre-to post-training was
examined for each training condition in order to test whether no-go
training led to the devaluation of no-go foods compared to control
foods. It should be noted that this was not a positive control for
training effectiveness, as the findings for stimulus devaluation
outcomes remain controversial (see Jones et al., 2016 for
meta-analysis). Stimulus devaluation in this study was therefore treated
both as a manipulation check for the employed training paradigm and a
secondary outcome measure.

\noindent H3. There will be moderate evidence for an effect of training
condition (go, no-go, control) on the change in food liking from pre-to
post-training.

\begin{itemize}
\item[H3a.] Participants will show reduced liking for no-go foods relative to the control foods, from pre-to post- training.
\item[H3b.] Participants will show increased liking for go foods relative to the control foods, from pre-to post- training.
\end{itemize}

\subsection{Manipulation check 2: Contingency
learning}\label{manipulation-check-2-contingency-learning}

Training performance was examined in terms of contingency learning. ICT
paradigms, such as the go/no-go training task, might lead to
stimulus-response associations and learning can be observed in the
reaction times and error rates for the different stimulus-response
mappings (e.g., N. S. Lawrence et al., 2015b). The percentage of
successful signal trials (i.e., successful stops) and the reaction times
from no-signal (go) trials were compared for specific training
conditions, as stated in the hypotheses below.

\noindent H4. Go/no-go training will result in contingency learning in
terms of reaction times on no-signal trials and the percentage of
successful inhibitions on signal trials.

\begin{itemize}
\item[H4a.] Percentage of successful stops will be greater for no-go foods compared to the control foods associated with a signal (control$_{nogo}$).
\item[H4b.] Go reaction times will be faster for go foods compared to the no-signal control foods (control$_{go}$).
\end{itemize}

\section{Methods}\label{methods}

\subsection{Participants}\label{participants}

\par

255 participants were recruited in total from the University campuses of
Cardiff, Bath and Exeter via research participation schemes (e.g.,
Experimental Management system; EMS) and advertisements (see Figure A1
for recruitment details). Participants recruited through participation
schemes received course credits, whereas other individuals were offered
entry into a prize draw for one of three £20 shopping vouchers.
Participants were informed about the study eligibility criteria and in
order to ensure compliance they completed a screening survey in the
beginning of the study and provided their consent. They were asked to
refrain from eating for 3 hours before the study. Participants had to be
at least 18 years of age, be fluent in spoken and written English and
have normal or corrected-to-normal vision, including normal colour
vision. Participants were excluded if they were dieting at the time of
the study, with a weight goal and time-frame in mind, had a current
and/or past diagnosis of any eating disorder(s) and had a
body-mass-index (BMI) lower than 18.5 kg/m\(^{2}\) (i.e., underweight
category). The study was approved by the Ethics Committees of Cardiff
University, University of Bath and the University of Exeter.

\subsection{Sampling plan}\label{sampling}

\par

The required sample size was estimated based on a frequentist power
analysis conducted for the primary outcome measure (i.e., change in
approach-avoidance bias, from pre-to post-training, between go and no-go
foods; H1a and H1b) and the stimulus devaluation manipulation check
(i.e., change in food liking, from pre-to-post training, between go and
no-go foods; H3). Both of these effect sizes were in the medium range,
we therefore based our calculations on the primary outcome measure. For
an expected effect size we considered other studies that have measured
approach bias pre-and post-approach-avoidance training (D. Becker,
Jostmann, Wiers, \& Holland, 2015; Schumacher, Kemps, \& Tiggemann,
2016). Both studies reported an effect size of
\textit{$\eta$$_{p}$$^{2}$}=0.07 which corresponds to a \enquote{medium}
effect size. D. Becker et al. (2015) also reported two non-significant
results, although effect sizes were not provided. Note, however, that D.
Becker et al. (2015) compared an active group with 90:10 mapping (i.e.,
avoidance of 90\% for unhealthy trials and 10\% healthy trials) to a
control group with 50:50 mapping whereas Schumacher et al. (2016)
compared a 90:10 active group with a 10:90 control group. We therefore
took a conservative approach when calculating our sample size. Firstly,
we reduced the effect size by 33\% (i.e., \emph{dz} = 0.34) to account
for publication bias (Button et al., 2013) and secondly we used an alpha
of .005, which has recently been recommended for any research that
cannot be considered a direct replication and can increase the
reliability of new discoveries (Benjamin et al., 2018). Based on a
priori power calculations using G*Power (Faul, Erdfelder, Buchner, \&
Lang, 2009) we estimated that a total sample of 149
participants\footnote{Due to the large number of participant exclusions based on mean error rates in the AAT (see Figure A1) and the group testing laboratory setting at Cardiff University, final recruitment led to the expected sample size including 14 more participants (N=163).}
was necessary for 90\% power.

\par  

The sampling method and power analysis of the study followed a
conservative frequentist approach, but the pre-registered analyses were
based on a Bayesian framework (see \textit{\nameref{prereg_analyses}}).
Frequentist analyses were also reported in a supplementary fashion
(\(\alpha\) = .005). Bayes factors (BFs) informed the interpretations of
the results and although debate exists about labelling evidence in terms
of BFs (Morey, 2015), we followed the guidelines by (Lee \& Wagenmakers,
2013). A threshold of \emph{BF}\(_{10}\) \textgreater{} 6 was used to
indicate moderate evidence for the alternative hypothesis relative to
the null, and \emph{BF}\(_{10}\) \textless{} 1/6 reflected moderate
evidence for the null relative to the respective alternative hypothesis.
Bayes factor analyses were favoured for drawing conclusions from the
study, as they would allow us to interpret null outcomes as evidence of
absence when traditional analyses would not make such inferences
feasible. For frequentists analyses, an alpha level of 0.005 was used.

\subsection{Procedure}\label{procedure}

The study procedure can be seen in Figure \ref{fig:procedure} (panel A).
After screening, eligible participants were provided with a short survey
(see \textit{\nameref{survey_questionnaires}}) and proceeded to rate all
food categories on how much they like the taste (see
\textit{\nameref{food_ratings}}). Three blocks of the approach-avoidance
task (AAT) were completed before the go/no-go training paradigm was
performed. Rated food categories wer randomly assigned to three
conditions for training: go, no-go and control, as shown in Figure
\ref{fig:procedure} (panel B). Post-training, participants were
presented with another three blocks of the AAT, provided ratings for all
food stimuli again and finally completed a short food choice task (see
\textit{\nameref{food_choice}}). At the end of the study, several
questionnaires were presented in random order (see
\textit{\nameref{survey_questionnaires}}) and participants were
debriefed about the aims of the study.

\begin{figure} [!htb]
\centering
\includegraphics[width=\linewidth]{figures/Figure1.png}
\caption{\textbf{Schematic diagram of the study procedure, go/no-go training and approach-avoidance tasks.} \textbf{A.} After completing the screening and initial survey, participants rated all food stimuli (liking) and proceeded to perform the pre-training approach-avoidance task (AAT) blocks. In the training phase, participants completed six blocks of go/no-go training. The post-training AAT blocks were then presented and followed by food liking ratings. At the end of the study, participants completed a short food choice task and several questionnaires, in random order. \textbf{B.} The go/no-go training paradigm involved go (no-signal) and no-go (signal) trials that occured with equal probability. On go trials, participants had to respond within 1250ms by pressing the "C" and "M" keys to indicate the picture location (left or right, respectively). On no-go trials, participants were instructed not to respond at all. The inter-trial interval (ITI) was 1250ms. Food categories were randomly assigned to three conditions. Go foods were only paired with no-signal trials and no-go foods were always associated with no-signal trials. Control, or filler, foods were presented in both signal and no-signal trials (50:50).}
\label{fig:procedure}
\end{figure}

\clearpage

\begin{figure}
    \ContinuedFloat
    \captionsetup{labelformat=empty}
    \caption{\textbf{C.} In the AAT, participants were asked to respond according to the format of the presented rectangle (portrait or landscape). Response-format assignements were approximately counterbalanced across participants. As an example, on approach trials a participant would have to pull the mouse towards them when the picture was in portrait format (approach trial) and push it away from them when the picture was in landscape format. Push and pull actions were paired with visual feedback, that is, zoom-out and zoom-in effects respectively. The maximum reaction time (maxRT) was 1500ms and the ITI was set to 500ms. Participants clicked on a central "X" to begin a trial (self-timed start).}
\end{figure}

\subsection{Go/No-go training}\label{gng}

\par

The Go/No-Go (GNG) training paradigm involved go and no-go responses to
six pre-selected appetitive food categories. Food categories differed in
terms of taste, so that three foods were savoury (i.e., pizza, crisps,
chips) and three foods were sweet (i.e., biscuits, chocolate,
cake)\footnote{All study materials are openly available at https://osf.io/wcf4r/}.
Two food categories were randomly assigned to each training condition
(go, no-go, filler foods) in the beginning of the experiment and food
taste was counterbalanced so that each condition had one sweet and one
savoury food. There were three training conditions according to the
mapping of foods to signal (no-go) and no-signal (go) trials in the GNG.
All go foods appeared in go (no-signal) trials and all no-go foods were
presented in no-go (signal) trials (see Figure 1, panel C). Control, or
filler, foods appeared on both go and no-go trials with equal
probability (i.e., 50:50). Each food category had three exemplars which
appeared twice in each block.

\par

All foods were presented on either the left or right hand side of the
screen within a rectangle for 1250ms, which was the maximum reaction
time (maxRT), as shown in Figure \ref{fig:procedure}, panel B.
Participants were asked to respond to the location of the food as
quickly and as accurately as possible by pressing the \enquote{C} and
\enquote{M} buttons on the keyboard with their left and right index
fingers, respectively. The central rectangle remained on the screen
throughout the training, including the inter-trial-interval (ITI), which
was 1250ms. On signal trials, the rectangle turned bold, indicating that
participants should withhold their response. In line with the GNG
training paradigm, this signal appeared on stimulus onset (i.e., no
delay between stimulus and signal) and stayed on the screen until the
end of the trial. A correct response on no-signal trials was registered
when participants responded accurately to the location of the food
within the maxRT window and a successful stop (i.e., correct signal
trial) was considered when participants did not respond at all.
Incorrect responses in no-signal trials refer to either to a wrong
location judgment or a missed response. Left and right responses were
counterbalanced across all manipulated variables for each type of trial.
Training was split into 6 blocks of 36 trials (216 trials in total) and
lasted approximately 10 minutes with inter-block breaks (15s). Task
practice included 12 trials of go and no-go responses (50\%-50\%) and
participants responded to the location of grey squares, instead of food
pictures. For the practice trials, we provided accuracy feedback during
the ITI.

\subsection{Approach avoidance task}\label{aat}

\par

The approach-avoidance task (AAT) was adapted from an existent paradigm
(M. Rinck \& Becker, 2007; Wiers, Rinck, Dictus, \& Van Den Wildenberg,
2009a), which involves \enquote{pull} (i.e., towards self) and
\enquote{push} (i.e, away from self) movements of a joystick. Each type
of motor response is paired with visual feedback so that when the
joystick is pulled, the image gets bigger (zoom-in) and when it is
pushed, the image gets smaller (zoom-out). This \enquote{zooming}
effects acts as an exteroceptive cue of either an approach or avoidance
response (Neumann \& Strack, 2000). This feature of the joystick AAT
complements the proprioceptive properties of the task, where responses
requiring arm flexion and extension correspond to approach and avoidance
trials, respectively. This task also disambiguates approach and
avoidance responses by using the \enquote{zooming} feature (Wiers et
al., 2009a). For example, arm extension could indicate an approach
response towards an appetitive food (object-reference) or an avoidance
response where the food is pushed away from the body/self
(self-reference; Phaf, Mohr, Rotteveel, \& Wicherts, 2014). The visual
feedback thus provides the self-reference attribute to the responses
(e.g., object comes closer to one's body). We also adopted the
evaluation-irrelevant feature of the paradigm, whereby participants
respond according to the format of (portrait or landscape; e.g., Wiers,
Rinck, Kordts, Houben, \& Strack, 2010a).

\par

AAT responses involved \enquote{push} and \enquote{pull} movements of
the computer mouse. Food stimuli were presented in the centre of the
screen and participants were instructed to pull the mouse towards them
or push the mouse away from them according to whether the image was in
portrait or landscape format (see Figure \ref{fig:procedure}).
Response-format assignments were approximately counterbalanced across
participants (45.4\% portrait-approach, 54.6\% landscape-approach).
Instructions highlighted moving the mouse cursor until it reaches the
end of the screen (top or bottom edge) for a correct response to be
registered and making smooth whole-arm movements. Participants had
1500ms to respond after the stimulus appeared. Each trial started with a
central \enquote{X} on the screen and participants had to click on it to
begin. The ITI was 500 ms and there was no delay between the \enquote{X}
click response and the stimulus onset. In order to account for the
natural movement of the mouse, pixel tolerance was added to every mouse
movement (\(\pm\) 1.25\% of display height), including movement
initiation in the beginning of the trial. A response in the AAT was
registered as correct only when participants completed the correct
action (e.g., pull or push) within the maxRT window and also initiated a
movement towards the correct direction. Even if the final response was
correct, participants could have changed their movement after making an
initial error (e.g., pull instead of push the mouse in an
\enquote{avoid} trial) and therefore the direction of their initial
movement was also taken into account. The complete RT for an AAT trial
was defined as the time from the stimulus onset to the successful
completion of a response.

\par

Each AAT block consisted of 72 trials and go, no-go and control foods
appeared with equal probability for both \enquote{pull} (approach) and
\enquote{push} (avoid) responses. There were 12 approach and 12 avoid
trials for each training condition (e.g., no-go) and within those
trials, there were six savoury and six sweet foods presented (i.e., 3
exemplars repeated twice). Three AAT blocks were performed before
training (AAT\(_{pre}\)) and three after training (AAT\(_{post}\)).
There was a number of constraints placed on the quasi-random order of
the trials within an AAT block. There were no more than three images of
the same food category being presented consecutively and no more than
three trials with the same picture format in sequence. AAT practice
consisted of 10 trials, whereby grey rectangles appeared in either
landscape or portrait format. Feedback was presented for practice trials
only. The screen background throughout the AAT was black and the task
lasted approximately 15 minutes, including the inter-block 15 second
breaks. Participants received a reminder of the instructions after each
inter-block 15 sec break.

\subsection{Food liking ratings}\label{food_ratings}

\par

Participants provided food liking ratings before and after training
using a visual analogue scale (VAS). Participants rated all foods
included in the GNG paradigm according to how much they liked the taste,
ranging from 0 (\enquote{not at all}) to 100 (\enquote{very much}). Task
instructions encouraged participants to imagine they were tasting the
food in their mouth and then rate how much they liked the taste. The
order of the presented foods was randomised and each block consisted of
18 trials. Participants completed a block before training
(Liking\(_{pre}\)) and a block post training (Liking\(_{post}\)).

\subsection{Food choice task}\label{food_choice}

Impulsive food choices were assessed using a food choice task adapted
from Veling et al. (2013), which included all food categories from the
GNG paradigm (two exemplars per category). The twelve foods were
presented in a grid layout on the screen and participants had ten
seconds to select three foods that they would like to consume the most
at that specific time, by clicking on them with the computer mouse.
Participants were asked to click on a \enquote{start} button to begin
the trial and when a response was registered the selected food stimulus
disappeared from the screen. We assumed that this task element would
prevent participants from deliberating on their choices and changing
their initial responses, which would mean that \textit{impulsive} food
choices were no longer measured. However, it shoudl be noted that
although participants were not informed about the hypothetical nature of
their choices, it is highly probably that they would not consider their
choices consequential (i.e., they would not think that would get a food
item after the task).

\subsection{Survey \& Questionnaires}\label{survey_questionnaires}

\par

Eligible participants were presented with an initial survey to record
demographics and other variables for exploratory analyses. The survey
constited of questions height and weight measurements to calculate
participant's body-mass-index (BMI; kg/m\textsuperscript{2}), the number
of hours since their last meal (\enquote{less than 3 hours ago},
\enquote{3-5 hours ago}, \enquote{5-10 hours ago}, \enquote{more than 10
hours ago}) and hunger state at the the time of the study
(VAS:1=\enquote{Not at all} to 9=\enquote{Very}). Gender was also
recorded with the options of male, female, transgender male, transgender
female, gender variant/non-conforming, and an open ended text response
for \enquote{other}.

\par

Several questionnaires were completed by the participants at the end of
the study for exploratory analyses, as part of the undergraduate student
projects of the GW4 Undergraduate Psychology Consortium 2017/2018. The
Barratt Impulsivity Scale (BIS-15; Spinella, 2007) was introduced to
explore the relationship between training outcomes and impulsivity. We
also examined a distinctive element of general trait self-control,
referred to as stop control, using the Stop Control Scale (SCS; De Boer,
van Hooft, \& Bakker, 2011). Other administered questionnaires included
the Food Cravings Questionnaire - Trait - reduced (FCQ-T-r; Meule,
Hermann, \& Kübler, 2014), Perceived Stress Scale (PSS; Cohen, Kamarck,
\& Mermelstein, 1983) and the \enquote{food} and \enquote{money}
subscales from the Delaying Gratification Inventory (DGI; Hoerger,
Quirk, \& Weed, 2011).

\section{Analyses}\label{analyses}

\subsection{Measures \& indices}\label{measures-indices}

\par

Reaction time and accuracy data from the GNG blocks were recorded for
all design cells for an exploratory assessment of training performance
(see \hl{[section]}). The mean error rates in no-signal and signal
trials as well as mean reaction time in no-signal trials (GoRT) informed
participant exclusions from all analyses (see
\textit{\nameref{data_exclusions}}). For the contingency learning
manipulation check (H3, H4), we measured the average proportion of
successful stops from signal trials for no-go foods and control foods
which were paired with a signal (controlnogo) and the mean GoRTs for
each participant from go and controlgo trials. \par
Performance in the AAT\(_{pre}\) and AAT\(_{post}\) blocks was
considered only for correct responses. We calculated median RTs for
\enquote{push} and \enquote{pull} responses on all training condition
levels, at a participant
level\footnote{RTs were recorded continuously from movement initiation to response completion with samples every 33ms (two display refresh rates) to allow dynamic zoom-in/zoom-out effects based on participants' mouse movements. However, a bug was encountered with the version of the software and the temporal resolution at which coordinates and times were recorded was reduced. For this reason, we used linear interpolation to increase our samples to 100 for every trial and obtain more precise RT measures. All details regarding this procedure and the software bug can be found in our analyses scripts.}.
Medians were used instead of means as they are less sensitive to
outliers in RT distributions and in line with previous literature
(Wiers, Rinck, Dictus, \& Van Den Wildenberg, 2009b; Wiers, Rinck,
Kordts, Houben, \& Strack, 2010b). The approach-avoid bias score for
each condition was calculated as the difference between the median RTs
for \enquote{push} and' pull' responses (MedianRT\(_{push}\)-
MedianRT\(_{pull}\)). Bias scores were computed for both AAT\(_{pre}\)
and AAT\(_{post}\) blocks. Positive scores indicate an approach bias
towards the foods of interest and negative scores reflect avoidance for
those foods. Change scores for approach-avoid biases from pre-to
post-training (\(\Delta\)AAT bias score) were calculated for
pre-registered analyses (H1). The proportion of correct responses for
each AAT design cell informed participant exclusions. \par
Participants were required to choose three foods out of twelve in the
food choice task and selections could vary in their number for each
training condition (go, no-go, control). Food choices were therefore
normalised according to the total number of responses per participant
(i.e., proportion). For the analyses of food choices, we compared the
mean proportions of choices (calculated per participant relative to
their total number of choices) between each training condition. Food
rating VAS scores were averaged (mean) across the two foods per training
condition (i.e., sweet and savoury foods for go, no-go and control
conditions) and the three exemplars of each food. Changes in food liking
were examined in terms of change scores (\(\Delta\)Food liking score)
from pre-to-post training.

\subsection{Data exclusions}\label{data_exclusions}

\par

Participant-level data exclusions were conducted based on GNG training
and AAT performance. Participants who met any of the following criteria
were excluded from all respective analyses. We excluded participants who
had a mean GoRT greater than three standard deviations from the group
mean and percentage of correct responses in no-signal trials less than
85\%. Participants were also excluded if their percentage of errors in
signal trials was greater than three standard deviations from the group
mean and percentage of errors in either pre- or post- AAT blocks greater
than 0.25. Additionally, participants who submitted a food rating of 50
(i.e., neutral) for 24 or more trials wither pre-or post-training would
not be included as we assumed that multiple such responses would
indicate that participants used the default setting of the VAS and
purposefully skipped the rating trials.

\subsection{Pre-registered analyses}\label{prereg_analyses}

\par

Data pre-processing and analyses were conducted in RStudio (RStudio
Team, 2016) and JASP (JASP Team, 2018). Pre-registered analyses are
described under their pre-specified hypotheses, as presented in
\textit{\nameref{hypotheses}}. Directional hypotheses will be tested via
Bayesian paired-samples t-tests (e.g., see H1a and H1b below).

\noindent H1. The effect of training condition on the change in
approach-avoid bias scores from pre-to post-training was examined using
a Bayesian Repeated Measures ANOVA with the default prior settings (J.
N. Rouder, Engelhardt, McCabe, \& Morey, 2016; J. N. Rouder, Morey,
Speckman, \& Province, 2012) and participants treated as a nuisance
term.

\begin{itemize}
\item[H1a.] $\Delta$AAT$_{nogo}$ <  $\Delta$AAT$_{control}$  
\item[H1b.] $\Delta$AAT$_{go}$ >  $\Delta$AAT$_{control}$
\end{itemize}

\noindent H2. Two Bayesian paired samples t-tests were conducted for the
mean proportions of selected foods in the go and no-go training
condition compared to the control.

\begin{itemize}
\item[H2a.] p(no-go) < p(control)
\item[H2b.] p(go) > p(control)
\end{itemize}

\noindent H3. The effect of training condition on the change in food
liking from pre-to post-training was examined using a Bayesian Repeated
Measures ANOVA, consistent with H1.

\begin{itemize}
\item[H3a.] $\Delta$Liking$_{nogo}$ < $\Delta$Liking$_{control}$
\item[H3b.] $\Delta$Liking$_{go}$ > $\Delta$Liking$_{control}$
\end{itemize}

\noindent H4. Contingency learning during go/no-go training was examined
using Bayesian paired-samples t-tests for the percentage of successful
inhibition trials and go reaction times.

\begin{itemize}
\item[H4a.] PCsignal$_{nogo}$ > PCsignal$_{control-nogo}$
\item[H4b.] GoRT$_{go}$ < GoRT$_{control-go}$
\end{itemize}

The evidential value of confirmatory findings was solely determined by
the Bayesian tests outlined in this section, as previously explained
(see \textit{\nameref{sampling_plan}}. Frequentist tests were conducted
for reporting purposes and further the reproducibility of findings
(e.g., potential use in meta-analyses). Frequentist paired samples
t-tests were two-tailed, in line with the reported power analysis and
for pairwise comparisons following the repeated measures ANOVAs for H1
and H3 were corrected for multiple comparisons (Bonferroni).
\hl{make sure we mention that frequentist tests will be two-tailed in line with our power analysis and what we pre-registered of course}

\section{Results}\label{results}

\subsection{Sample characteristics}\label{sample-characteristics}

The final sample for pre-registered analyses consisted of 163
participants (80.98\% female). Detailed participant-level exclusions are
presented in Figure A1. Participants had on average a healthy BMI
(\emph{M} = 22.88, \emph{SD} = 2.98, range = 18.54 - 32.36) and their
mean age was 22.39 (\emph{SD} = 9.04, range = 18-59). 108 participants
(66.26\%) reported that they had their last meal 3-5 hours before the
study and hunger levels at the beginning of the study were not
particularly high (\emph{M} = 5.70, \emph{SD} = 2.22). However, 24
participants (14.72\%) did not adhere to the instruction not to eat
three hours before the study, as they reported having their last meal
\enquote{less than 3 hours ago}.

\begin{table}[h]
    \centering
    \caption{Bayesian Pearson Correlations}
    \label{tab:bayesianPearsonCorrelations}
    {
        \begin{tabular}{lrrrrrrr}
            \toprule
             &  & 1. & 2. & 3. & 4. & 5. & 6.  \\
            \cmidrule[0.4pt]{1-8}
            1$.$ FCQ-T-r total & Pearson's r & -- &   &   &   &   &    \\
             & log(BF$_{1}$$_{0}$) & -- &  &  &  &  &   \\
            2$.$ BIS total & Pearson's r & 0.491 & -- &   &   &   &    \\
             & log(BF$_{1}$$_{0}$) & 19.572 & -- &  &  &  &   \\
            3$.$ PSS total & Pearson's r & 0.462 & 0.316 & -- &   &   &    \\
             & log(BF$_{1}$$_{0}$) & 16.754 & 6.043 & -- &  &  &   \\
            4$.$ SCS total & Pearson's r & -0.374 & -0.721 & -0.260 & -- &   &    \\
             & log(BF$_{1}$$_{0}$) & 9.630 & 56.042 & 3.247 & -- &  &   \\
            5$.$ DGI food subscale & Pearson's r & -0.612 & -0.433 & -0.226 & 0.376 & -- &    \\
             & log(BF$_{1}$$_{0}$) & 35.070 & 14.168 & 1.849 & 9.777 & -- &   \\
            6$.$ BMI (kg/m$^{2}$) & Pearson's r & 0.246 & 0.161 & 0.125 & -0.122 & -0.189 & --  \\
             & log(BF$_{1}$$_{0}$) & 2.655 & -0.245 & -1.067 & -1.133 & 0.577 & --  \\
            \bottomrule
            % \multicolumn{8}{p{0.5\linewidth}}}
        \end{tabular}
    }
    \begin{tablenotes}[para]
\footnotesize{\textit{Note.} * log(BF$_{1}$$_{0}$)  > log(10), **  log(BF$_{1}$$_{0}$)  > log(30), ***  log(BF$_{1}$$_{0}$)  > log(100)}
\end{tablenotes}
\end{table}

\subsection{Confirmatory findings for training
outcomes}\label{confirmatory-findings-for-training-outcomes}

\par

We found \emph{strong} evidence for the absence of a general effect of
go/no-go training condition on the change in approach-bias scores
{[}\emph{BF}\(_{01}\) = 15.99; \emph{F}(2, 324) = 1.01, \emph{p} =
0.365{]}. There was moderate evidence (\emph{BF}\(_{01}\) = 9.31) that
the change in bias scores for no-go foods (\(\Delta\)AAT\(_{nogo}\);
\emph{M} = -3.31, \emph{SD} = 62.91) was not reduced compared to the
change for filler foods (\(\Delta\)AAT\(_{control}\); \emph{M} = -1.81,
\emph{SD} = 59.55), as shown in Table \hl{add label here}. Similar to
H1a, we found \emph{strong} evidence (\emph{BF}\(_{01}\) = 25.73) for
the null compared to the alternative for H1b. The change in bias scores
for go foods (\(\Delta\)AAT\(_{go}\); \emph{M} = -10.47, \emph{SD} =
59.57) was not greater than the change for filler foods.

\par

The effect of training on impulsive food choices was examined for no-go
and go foods compared to control, as stated in H2a and H2b respectively.
There was \emph{extreme} evidence that the probability of choosing a
no-go food (\emph{M} = 0.21, \emph{SD} = 0.27) was reduced compared to
the probability of choosing a filler food (\emph{M} = 0.36; \emph{SD} =
0.31) after training {[}H2a; \emph{BF}\(_{10}\) = 247.78; \emph{t}(161)
= -3.93, \emph{p} \textless{} .001, \emph{d} = -0.31, 95\% CI for
\emph{d} = -0.47,
-0.15\footnote{Shapiro-Wilk tests suggested deviation from normality (\textit{p} < .001) and the results from the Wilcoxon signed-ranked tests are reported here. H2a: \textit{W} = 1875.00, \textit{p} < .001, \textit{r} = -0.716; H2b: \textit{W} = 4663.00, \textit{p} = 0.048, \textit{r} = -0.294}{]}.
In contrast, we only obtained \emph{anecdotal} evidence that probability
of choosing a go food (\emph{M} = 0.44; \emph{SD} = 0.33) was not
greater than the probability of choosing a filler food after training
{[}\emph{BF}\(_{01}\) = 1.18, \emph{t}(161) = 1.82, \emph{p} = 0.070,
\emph{d} = 0.14, 95\% CI for \emph{d} = -0.01, 0.30{]}.

\subsection{Manipulation checks for
training}\label{manipulation-checks-for-training}

\par

\hl{we pre-registered an anova too for this} In order to validate
whether the implemented go/no-go training paradigm led to
stimulus-response associations (i.e., contingency learning) we first
tested whether the percentage of correct responses for no-go foods
(i.e., successful inhibitions) would be greater compared to the
percentage of correct responses for filler foods associated with signal
trials (H4a). There was \emph{extreme} evidence that participants had on
average more successful inhibitions for no-go foods
(PCsignal\(_{nogo}\); \emph{M} = 0.97, \emph{SD} = 0.03) than filler
foods (PCsignal\(_{control-nogo}\); \emph{M} = 0.96, \emph{SD} = 0.04)
{[}\emph{BF}\(_{10}\) = 140.30, \hl{..put in a table?}{]} For H4b, we
tested whether mean reaction times would be reduced for go foods
(GoRT\(_{go}\); \emph{M} = 507.00, \emph{SD} = 70.48) compared to filler
foods associated with no-signal trials (GoRT\(_{control-go}\); \emph{M}
= 515.00, \emph{SD} = 75.51) and there was \emph{extreme} evidence for
such an effect. Therefore, contingency learning was observed in the
employed GNG paradigm for both reaction time and accuracy outcomes.

\par

As a second manipulation check for training outcomes, we investigated
whether GNG changed the evaluations of foods associated with signal and
no-signal trials compared to the evaluations of filler foods which were
paired with either type of trial with equal probability (control).
\hl{maybe point for discussion: if evaluations are already high we might expect to see a ceiling effect- as reported in chen's latest papers- what we didn't test for and might be worth looking at is no-go versus go! extra validation- can strengthn the claim that stimulus devaluation occured}

The change in liking scores from pre-to post-training for nogo foods
(\(\Delta\)Liking\(_{nogo}\); \emph{M} = -4.16; \emph{SD} = 9.51) was
only slightly reduced compared to change in liking for filler foods
(\(\Delta\)Liking\(_{control}\); \emph{M} = -2.61, \emph{SD} = 8.77),
and there was only anecdotal evidence for this effect {[}H3a;
\emph{BF}\(_{10}\) = 2.65, \emph{t}(162) = -2.38, \emph{p} = 0.019,
\emph{d} = -0.19, 95\% CI for \emph{d} = -0.34,
-0.03\footnote{Although data transformations or alternative tests were not pre-registered for the potential violation of the normality assumption, the Shapiro-Wilk test showed a deviation from normality (\textit{p} < .001) and thus we also report the results from the Wilcoxon signed-rank tests. The effect size is given by the matched rank biserial correlation (\textit{r}) H3a: \textit{W} = 5907.50, \textit{p} = 0.246, \textit{r} = -0.116; H3b: \textit{W} = 6666.50, \textit{p} = 0.914},
\textit{r} = -0.002{]}. The change in liking scores from pre-to
post-training for go foods (\(\Delta\)Liking\(_{go}\); \emph{M} = -2.87,
\emph{SD} = 10.15), however, was not greater than the change for filler
foods as originally expected.
\hl{this is not surprising as there may be a ceiling effect in how much these foods can be liked more.. see chen for discussion}
Instead, there was \emph{strong} evidence for the null hypothesis
compared to the alternative {[}H3b; \emph{BF}\(_{01}\) = 14.95,
\emph{t}(162) = -0.37, \emph{p} = 0.715, \emph{d} = -0.03, 95\% CI for
\emph{d} = -0.18, 0.13{]}.

\section{Discussion}\label{discussion}

\newpage

\include{Appendix}

\section{References}\label{references}

\begingroup
\setlength{\parindent}{-0.5in} \setlength{\leftskip}{0.5in}

\hypertarget{refs}{}
\hypertarget{ref-adams_training_2017}{}
Adams, R. C., Lawrence, N. S., Verbruggen, F., \& Chambers, C. D.
(2017). Training response inhibition to reduce food consumption:
Mechanisms, stimulus specificity and appropriate training protocols.
\emph{Appetite}, \emph{109}, 11--23.
\url{https://doi.org/10.1016/j.appet.2016.11.014}

\hypertarget{ref-allom_does_2016}{}
Allom, V., Mullan, B., \& Hagger, M. (2016). Does inhibitory control
training improve health behaviour? A meta-analysis. \emph{Health
Psychol. Rev.}, \emph{10}(2), 168--186.
\url{https://doi.org/10.1080/17437199.2015.1051078}

\hypertarget{ref-becker_approach_2015-1}{}
Becker, D., Jostmann, N. B., Wiers, R. W., \& Holland, R. W. (2015).
Approach avoidance training in the eating domain: Testing the
effectiveness across three single session studies. \emph{Appetite},
\emph{85}(June 2015), 58--65.
\url{https://doi.org/10.1016/j.appet.2014.11.017}

\hypertarget{ref-benjamin_redefine_2017}{}
Benjamin, D. J., Berger, J. O., Johannesson, M., Nosek, B. A.,
Wagenmakers, E.-J., Berk, R., \ldots{} Johnson, V. E. (2018). Redefine
statistical significance. \emph{Nature Human Behaviour}, \emph{2},
6--10. \url{https://doi.org/10.1038/s41562-017-0189-z}

\hypertarget{ref-button_power_2013}{}
Button, K. S., Ioannidis, J. P. A., Mokrysz, C., Nosek, B. A., Flint,
J., Robinson, E. S. J., \& Munaf\a`o, M. R. (2013). Power failure: Why
small sample size undermines the reliability of neuroscience. \emph{Nat.
Rev. Neurosci.}, \emph{14}(5), 365--376.
\url{https://doi.org/10.1038/nrn3475}

\hypertarget{ref-chen_when_2018}{}
Chen, Z., Holland, R., Quandt, J., Dijksterhuis, A., \& Veling, H.
(2018). When mere action versus inaction leads to robust preference
change. \url{https://doi.org/10.17605/OSF.IO/ZY9W3}

\hypertarget{ref-chen_how_2016-1}{}
Chen, Z., Veling, H., Dijksterhuis, A., \& Holland, R. W. (2016a). How
does not responding to appetitive stimuli cause devaluation: Evaluative
conditioning or response inhibition? \emph{Journal of Experimental
Psychology: General}, \emph{145}(12), 1687--1701.
\url{https://doi.org/10.1037/xge0000236}

\hypertarget{ref-chen_how_2016}{}
Chen, Z., Veling, H., Dijksterhuis, A., \& Holland, R. W. (2016b). How
does not responding to appetitive stimuli cause devaluation: Evaluative
conditioning or response inhibition? \emph{Journal of Experimental
Psychology: General}, \emph{145}(12), 1687--1701.
\url{https://doi.org/10.1037/xge0000236}

\hypertarget{ref-cohenGlobalMeasurePerceived1983}{}
Cohen, S., Kamarck, T., \& Mermelstein, R. (1983). A global measure of
perceived stress. \emph{Journal of Health and Social Behavior},
\emph{24}(4), 385--396.

\hypertarget{ref-deboerStopStartControl2011}{}
De Boer, B. J., van Hooft, E. A. J., \& Bakker, A. B. (2011). Stop and
start control: A distinction within self-control. \emph{European Journal
of Personality}, \emph{25}(5), 349--362.
\url{https://doi.org/10.1002/per.796}

\hypertarget{ref-faul_statistical_2009}{}
Faul, F., Erdfelder, E., Buchner, A., \& Lang, A.-G. (2009). Statistical
power analyses using G*Power 3.1: Tests for correlation and regression
analyses. \emph{Behav. Res. Methods}, \emph{41}(4), 1149--1160.
\url{https://doi.org/10.3758/BRM.41.4.1149}

\hypertarget{ref-hoergerDevelopmentValidationDelaying2011}{}
Hoerger, M., Quirk, S. W., \& Weed, N. C. (2011). Development and
validation of the Delaying Gratification Inventory. \emph{Psychological
Assessment}, \emph{23}(3), 725--738.
\url{https://doi.org/10.1037/a0023286}

\hypertarget{ref-houben_training_2011}{}
Houben, K., \& Jansen, A. (2011). Training inhibitory control. A recipe
for resisting sweet temptations. \emph{Appetite}, \emph{56}(2),
345--349. \url{https://doi.org/10.1016/j.appet.2010.12.017}

\hypertarget{ref-houben_chocolate_2015}{}
Houben, K., \& Jansen, A. (2015). Chocolate equals stop:
Chocolate-specific inhibition training reduces chocolate intake and go
associations with chocolate. \emph{Appetite}, \emph{87}, 318--323.
\url{https://doi.org/10.1016/j.appet.2015.01.005}

\hypertarget{ref-houben_too_2012}{}
Houben, K., Nederkoorn, C., \& Jansen, A. (2012). Too tempting to
resist? Past success at weight control rather than dietary restraint
determines exposure-induced disinhibited eating. \emph{Appetite},
\emph{59}(2), 550--555.
\url{https://doi.org/10.1016/j.appet.2012.07.004}

\hypertarget{ref-houben_eating_2014}{}
Houben, K., Nederkoorn, C., \& Jansen, A. (2014). Eating on impulse: The
relation between overweight and food-specific inhibitory control.
\emph{Obesity}, \emph{22}(5), 2013--2015.
\url{https://doi.org/10.1002/oby.20670}

\hypertarget{ref-JASP2018:1}{}
JASP Team. (2018). JASP (Version 0.10.0){[}Computer software{]}.

\hypertarget{ref-jones_inhibitory_2016}{}
Jones, A., Di Lemma, L. C., Robinson, E., Christiansen, P., Nolan, S.,
Tudur-Smith, C., \& Field, M. (2016). Inhibitory control training for
appetitive behaviour change: A meta-analytic investigation of mechanisms
of action and moderators of effectiveness. \emph{Appetite}, \emph{97},
16--28. \url{https://doi.org/10.1016/j.appet.2015.11.013}

\hypertarget{ref-kakoschke_combined_2015-1}{}
Kakoschke, N., Kemps, E., \& Tiggemann, M. (2015). Combined effects of
cognitive bias for food cues and poor inhibitory control on unhealthy
food intake. \emph{Appetite}, \emph{87}, 358--364.
\url{https://doi.org/10.1016/j.appet.2015.01.004}

\hypertarget{ref-lavagnino_inhibitory_2016}{}
Lavagnino, L., Arnone, D., Cao, B., Soares, J. C., \& Selvaraj, S.
(2016). Inhibitory control in obesity and binge eating disorder: A
systematic review and meta-analysis of neurocognitive and neuroimaging
studies. \emph{Neurosci. Biobehav. Rev.}, \emph{68}, 714--726.
\url{https://doi.org/10.1016/j.neubiorev.2016.06.041}

\hypertarget{ref-lawrence_nucleus_2012-1}{}
Lawrence, N. S., Hinton, E. C., Parkinson, J. A., \& Lawrence, A. D.
(2012). Nucleus accumbens response to food cues predicts subsequent
snack consumption in women and increased body mass index in those with
reduced self-control. \emph{NeuroImage}, \emph{63}(1), 415--422.
\url{https://doi.org/10.1016/j.neuroimage.2012.06.070}

\hypertarget{ref-lawrence_training_2015}{}
Lawrence, N. S., O'Sullivan, J., Parslow, D., Javaid, M., Adams, R. C.,
Chambers, C. D., \ldots{} Verbruggen, F. (2015a). Training response
inhibition to food is associated with weight loss and reduced energy
intake. \emph{Appetite}, \emph{95}, 17--28.
\url{https://doi.org/10.1016/j.appet.2015.06.009}

\hypertarget{ref-lawrenceTrainingResponseInhibition2015}{}
Lawrence, N. S., O'Sullivan, J., Parslow, D., Javaid, M., Adams, R. C.,
Chambers, C. D., \ldots{} Verbruggen, F. (2015b). Training response
inhibition to food is associated with weight loss and reduced energy
intake. \emph{Appetite}, \emph{95}, 17--28.
\url{https://doi.org/10.1016/j.appet.2015.06.009}

\hypertarget{ref-lawrence_stopping_2015-3}{}
Lawrence, N. S., Verbruggen, F., Morrison, S., Adams, R. C., \&
Chambers, C. D. (2015a). Stopping to food can reduce intake. Effects of
stimulus-specificity and individual differences in dietary restraint.
\emph{Appetite}, \emph{85}, 91--103.
\url{https://doi.org/10.1016/j.appet.2014.11.006}

\hypertarget{ref-lawrence_stopping_2015}{}
Lawrence, N. S., Verbruggen, F., Morrison, S., Adams, R. C., \&
Chambers, C. D. (2015b). Stopping to food can reduce intake. Effects of
stimulus-specificity and individual differences in dietary restraint.
\emph{Appetite}, \emph{85}, 91--103.
\url{https://doi.org/10.1016/j.appet.2014.11.006}

\hypertarget{ref-lee_bayesian_2013}{}
Lee, M. D., \& Wagenmakers, E.-J. (2013). \emph{Bayesian Cognitive
Modeling: A Practical Course}. Cambridge University Press.
\url{https://doi.org/10.1017/CBO9781139087759}

\hypertarget{ref-meule_short_2014}{}
Meule, A., Hermann, T., \& Kübler, A. (2014). A short version of the
Food Cravings QuestionnaireTrait: The FCQ-T-reduced. \emph{Frontiers in
Psychology}, \emph{5}. \url{https://doi.org/10.3389/fpsyg.2014.00190}

\hypertarget{ref-richard_d._morey_verbal_2015}{}
Morey, R. D. (2015). On verbal categories for the interpretation of
Bayes factors.

\hypertarget{ref-nederkoorn_specificity_2012}{}
Nederkoorn, C., Coelho, J. S., Guerrieri, R., Houben, K., \& Jansen, A.
(2012). Specificity of the failure to inhibit responses in overweight
children. \emph{Appetite}, \emph{59}(2), 409--413.
\url{https://doi.org/10.1016/j.appet.2012.05.028}

\hypertarget{ref-neumann_approach_2000}{}
Neumann, R., \& Strack, F. (2000). Approach and Avoidance: The Influence
of Proprioceptive and Exteroceptive Cues on Encoding of Affective
Information. \emph{J. Personal. Soc. Psychol.}, \emph{79}(1), 39--48.
\url{https://doi.org/10.1037//0022-3514.79.1.39}

\hypertarget{ref-phaf_approach_2014}{}
Phaf, R. H., Mohr, S. E., Rotteveel, M., \& Wicherts, J. M. (2014).
Approach, avoidance, and affect: A meta-analysis of approach-avoidance
tendencies in manual reaction time tasks. \emph{Front. Psychol.},
\emph{5}(378), 1--16. \url{https://doi.org/10.3389/fpsyg.2014.00378}

\hypertarget{ref-rinck_approach_2007}{}
Rinck, M., \& Becker, E. S. (2007). Approach and avoidance in fear of
spiders. \emph{Journal of Behavior Therapy and Experimental Psychiatry},
\emph{38}(2), 105--120.
\url{https://doi.org/10.1016/j.jbtep.2006.10.001}

\hypertarget{ref-rouderModelComparisonANOVA2016}{}
Rouder, J. N., Engelhardt, C. R., McCabe, S., \& Morey, R. D. (2016).
Model comparison in ANOVA. \emph{Psychon. Bull. Rev.}, \emph{23}(6),
1779--1786. \url{https://doi.org/10.3758/s13423-016-1026-5}

\hypertarget{ref-rouderDefaultBayesFactors2012}{}
Rouder, J. N., Morey, R. D., Speckman, P. L., \& Province, J. M. (2012).
Default Bayes factors for ANOVA designs. \emph{Journal of Mathematical
Psychology}, \emph{56}(5), 356--374.
\url{https://doi.org/10.1016/j.jmp.2012.08.001}

\hypertarget{ref-rstudio}{}
RStudio Team. (2016). \emph{RStudio: Integrated Development Environment
for R}. Boston, MA: RStudio, Inc.

\hypertarget{ref-schumacher_bias_2016}{}
Schumacher, S. E., Kemps, E., \& Tiggemann, M. (2016). Bias modification
training can alter approach bias and chocolate consumption.
\emph{Appetite}, \emph{96}, 219--224.
\url{https://doi.org/10.1016/j.appet.2015.09.014}

\hypertarget{ref-spinellaNormativeDataShort2007}{}
Spinella, M. (2007). Normative Data and a Short Form of the Barratt
Impulsiveness Scale. \emph{International Journal of Neuroscience},
\emph{117}(3), 359--368. \url{https://doi.org/10.1080/00207450600588881}

\hypertarget{ref-strack_reflective_2004}{}
Strack, F., \& Deutsch, R. (2004). Reflective and Impulsive Determinants
of Social Behavior. \emph{Personality and Social Psychology Review},
\emph{8}(3), 28.

\hypertarget{ref-veling_using_2011}{}
Veling, H., Aarts, H., \& Papies, E. K. (2011). Using stop signals to
inhibit chronic dieters' responses toward palatable foods. \emph{Behav.
Res. Ther.}, \emph{49}(11), 771--780.
\url{https://doi.org/10.1016/j.brat.2011.08.005}

\hypertarget{ref-veling_stop_2013}{}
Veling, H., Aarts, H., \& Stroebe, W. (2013). Stop signals decrease
choices for palatable foods through decreased food evaluation.
\emph{Front. Psychol.}, \emph{4}(875), 1--7.
\url{https://doi.org/10.3389/fpsyg.2013.00875}

\hypertarget{ref-veling_training_2017-1}{}
Veling, H., Chen, Z., Tombrock, M. C., M. Verpaalen, I. a., Schmitz, L.
I., Dijksterhuis, A., \& Holland, R. W. (2017). Training Impulsive
Choices for Healthy and Sustainable Food. \emph{J. Exp. Psychol. Appl.},
\emph{23}(1), 1--14. \url{https://doi.org/10.1037/xap0000112}

\hypertarget{ref-veling_when_2008}{}
Veling, H., Holland, R. W., \& van Knippenberg, A. (2008). When approach
motivation and behavioral inhibition collide: Behavior regulation
through stimulus devaluation. \emph{Journal of Experimental Social
Psychology}, \emph{44}(4), 1013--1019.
\url{https://doi.org/10.1016/j.jesp.2008.03.004}

\hypertarget{ref-veling_what_2017}{}
Veling, H., Lawrence, N. S., Chen, Z., van Koningsbruggen, G. M., \&
Holland, R. W. (2017). What Is Trained During Food Go/No-Go Training? A
Review Focusing on Mechanisms and a Research Agenda. \emph{Curr. Addict.
Reports}, \emph{4}(1), 35--41.
\url{https://doi.org/10.1007/s40429-017-0131-5}

\hypertarget{ref-veling_targeting_2014}{}
Veling, H., van Koningsbruggen, G. M., Aarts, H., \& Stroebe, W. (2014).
Targeting impulsive processes of eating behavior via the internet.
Effects on body weight. \emph{Appetite}, \emph{78}, 102--109.
\url{https://doi.org/10.1016/j.appet.2014.03.014}

\hypertarget{ref-wiers_relatively_2009}{}
Wiers, R. W., Rinck, M., Dictus, M., \& Van Den Wildenberg, E. (2009a).
Relatively strong automatic appetitive action-tendencies in male
carriers of the OPRM1 G-allele. \emph{Genes, Brain Behav.}, \emph{8}(1),
101--106. \url{https://doi.org/10.1111/j.1601-183X.2008.00454.x}

\hypertarget{ref-wiersRelativelyStrongAutomatic2009}{}
Wiers, R. W., Rinck, M., Dictus, M., \& Van Den Wildenberg, E. (2009b).
Relatively strong automatic appetitive action-tendencies in male
carriers of the OPRM1 G-allele. \emph{Genes, Brain Behav.}, \emph{8}(1),
101--106. \url{https://doi.org/10.1111/j.1601-183X.2008.00454.x}

\hypertarget{ref-wiers_retraining_2010}{}
Wiers, R. W., Rinck, M., Kordts, R., Houben, K., \& Strack, F. (2010a).
Retraining automatic action-tendencies to approach alcohol in hazardous
drinkers. \emph{Addiction}, \emph{105}(2), 279--287.
\url{https://doi.org/10.1111/j.1360-0443.2009.02775.x}

\hypertarget{ref-wiersRetrainingAutomaticAction2010}{}
Wiers, R. W., Rinck, M., Kordts, R., Houben, K., \& Strack, F. (2010b).
Retraining automatic action-tendencies to approach alcohol in hazardous
drinkers. \emph{Addiction}, \emph{105}(2), 279--287.
\url{https://doi.org/10.1111/j.1360-0443.2009.02775.x}

\endgroup --\textgreater{}


\end{document}
